\documentclass[a4paper,12pt,reqno]{article}

\usepackage{styledoc19}
\addbibresource{references.bib}

\definecolor{dkgreen}{rgb}{0,0.6,0}
\definecolor{gray}{rgb}{0.5,0.5,0.5}
\definecolor{mauve}{rgb}{0.58,0,0.82}

\lstset{frame=tb,
  language=Java,
  aboveskip=3mm,
  belowskip=3mm,
  showstringspaces=false,
  columns=flexible,
  basicstyle={\small\ttfamily},
  numbers=none,
  numberstyle=\tiny\color{gray},
  keywordstyle=\color{blue},
  commentstyle=\color{dkgreen},
  stringstyle=\color{mauve},
  breaklines=true,
  breakatwhitespace=true,
  tabsize=3
}


\begin{document} % конец преамбулы, начало документа
    \docNumber{RU.17701729.12.20-01 ТП 01-1}
    \docFormat{Текст программы}
    \student{БПИ 213}{В. А. Андронов}
    \project{Веб-приложение для ведения лабораторных и вычислительных экспериментов с поддержкой онтологий\unskip}
    \supervisor{Доцент департамента \vfill программной инженерии \vfill факультета компьютерных \vfill наук, кандидат \vfill технических наук}{О. В. Максименкова}

    \firstPage
    \newpage
    \secondPage
    \newpage
    \thirdPage
    \newpage

    \anonsubsection{Структура приложения}

    В данном документе представлено описание структуры репозитория, в котором
    находится исходный код программы «Веб-приложение для ведения лабораторных и вычислительных экспериментов с поддержкой онтологий\unskip».

    Репозиторий программы находится по ссылке:
    \hyperlink{https://github.com/AndronovVladislav/Diploma-ELN}{https://github.com/AndronovVladislav/Diploma-ELN}

    Полный текст не приводится в связи с большим объёмом кодовой базы.

    Основная программа находится в папке \texttt{backend} и \texttt{frontend}:
    
    \begin{enumerate}
        \item В папке \texttt{backend/alembic} содержатся настройки инструмента Alembic и миграции базы данных.
        \item В папке \texttt{backend/models} содержатся конфигурация подключения к базе данных и ORM-модели базы данных, включая эксперименты, пользователей и онтологии, а также код для работы с онтологиями.
        \item В папке \texttt{backend/routes} содержатся API-маршруты для взаимодействия с клиентской частью.
        \item В папке \texttt{backend/services} реализована бизнес-логика приложения.
        \item В папке \texttt{backend/schemas} хранятся pydantic-схемы для валидации входных и выходных данных API.
        \item В папке \texttt{backend/tests} хранятся юнит и интеграционные тесты.
        \item В файле \texttt{backend/config.py} находится конфигурация окружения и настройки приложения.
        \item В файле \texttt{backend/main.py} содержится основная точка входа в серверное приложение.
    \end{enumerate}
    
    Фронтенд приложения расположен в папке \texttt{frontend}:

    \begin{enumerate}
        \item В папке \texttt{frontend/src/views} содержатся представления страниц и интерфейсов.
        \item В папке \texttt{frontend/src/components} содержатся переиспользуемые UI-компоненты.
        \item В папке \texttt{frontend/src/stores} содержатся Pinia-хранилища для управления состоянием приложения.
        \item В папке \texttt{frontend/src/router} находится конфигурация маршрутов.
        \item В папке \texttt{frontend/src/api} содержится настройка Axios для взаимодействия с API.
        \item В файле \texttt{frontend/src/main.ts} находится точка входа в клиентское приложение.
    \end{enumerate}

    Прочие ресурсы программы:

    \begin{enumerate}
        \item В файлах \texttt{docker-compose.*.yml} содержится конфигурация контейнеров для развертывания.
        \item В файле \texttt{.env} хранятся переменные окружения для запуска Docker Compose.
        \item В файле \texttt{frontend/package.json} содержатся зависимости и скрипты для фронтенда.
        \item В файле \texttt{backend/pyproject.toml} содержится список зависимостей и настройки инструментов для бэкенда.
    \end{enumerate}
    
    \newpage
    
    \subsection*{Внешние зависимости}
    
    В проекте используются следующие внешние библиотеки и фреймворки:
    
    \begin{enumerate}
        \item FastAPI — веб-фреймворк для серверной части~\cite{Framework:FastAPI}.
        \item SQLAlchemy — ORM для работы с базой данных~\cite{Library:SQLAlchemy}.
        \item Alembic — инструмент управления миграциями~\cite{Library:Alembic}.
        \item Pydantic — валидация данных в API~\cite{Library:Pydantic}.
        \item Neo4j — графовая база данных для хранения онтологий~\cite{DB:Neo4j}.
        \item NeoSemantics — интеграция Neo4j с RDF и семантическими данными~\cite{Library:NeoSemantics}.
        \item Axios — библиотека для отправки HTTP-запросов во фронтенде~\cite{Library:Axios}.
        \item Vue.js — фреймворк для клиентской части~\cite{Framework:VueJS}.
        \item PrimeVue — набор UI-компонентов~\cite{Framework:PrimeVue}.
        \item Pinia — система управления состоянием~\cite{Library:Pinia}.
        \item Docker — контейнеризация приложения~\cite{Tool:Docker}.
        \item Docker Compose — управление многоконтейнерными приложениями~\cite{Tool:DockerCompose}.
        \item Nginx — обратный прокси-сервер для маршрутизации запросов~\cite{Tool:Nginx}.
        \item GitHub Actions — CI/CD для автоматизированного развертывания~\cite{Tool:GitHubActions}.
        \item Pytest — тестирование серверной части~\cite{Library:Pytest}.
        \item Tailwind CSS — фреймворк для стилизации интерфейсов~\cite{Library:TailwindCSS}.
        \item Cypher Query Language — язык запросов для Neo4j~\cite{QueryLang:CypherQL}.
    \end{enumerate}
    \newpage
    \printbibliography[title=Список источников, heading=bibintoc]

    \newpage
    \listRegistration

\end{document} % конец документа