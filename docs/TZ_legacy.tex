\documentclass[a4paper,12pt,reqno]{article}

\usepackage{styledoc19}

\begin{document}

    \docNumber{RU.17701729.12.20-01 ТЗ 01-1}
    \docFormat{Техническое задание}
    \student{БПИ 213}{В. А. Андронов}
    \project{Веб-приложение для ведения лабораторных и вычислительных экспериментов с поддержкой онтологий\unskip}
    \supervisor{Доцент департамента \vfill программной инженерии}{О. В. Максименкова}

    \firstPage
    \newpage

    \secondPage
    \newpage

    \thirdPage
    \newpage


    \section{Введение}

    \subsection{Наименование программы}

    \subsubsection{Наименование программы на русском языке}
    Веб-приложение для ведения лабораторных и вычислительных экспериментов с поддержкой онтологий\unskip

    \subsubsection{Наименование программы на английском языке}
    Ontologically-controlled data management of laboratory and computational experiments in chemistry and materials science\unskip

    \subsection{Краткая характеристика области применения}
    Онтологически-контролируемое управление данными представляет собой современный подход к систематизации и обработке информации в химии и материаловедении. Данный метод основывается на использовании онтологий — формальных описаний концепций и их взаимосвязей в конкретной предметной области, что позволяет создать единое семантическое пространство для обмена знаниями и данными \cite{ontology:base}.

    В условиях увеличивающихся объемов данных и сложных междисциплинарных исследований, необходимость в эффективных инструментах управления информацией становится все более очевидной. Онтологически-контролируемое управление данными позволяет не только организовывать большие объемы информации, но и обеспечивать их интероперабельность. Это упрощает процессы интеграции данных из различных источников и способствует более тесному сотрудничеству между научными коллективами и организациями, занимающимися исследованиями в смежных областях.
    \newpage


    \section{Основания для разработки}

    \subsection{Документы, на основании которых ведется разработка}
    Приказ декана факультета компьютерных наук И.В. Аржанцева <<Об утверждении тем, руководителей выпускных квалификационных работ студентов образовательной программы «Программная инженерия» факультета компьютерных наук>> № X от DD.MM.YYYY.

    \subsection{Наименование темы разработки}
    Наименование темы разработки – <<Веб-приложение для ведения лабораторных и вычислительных экспериментов с поддержкой онтологий\unskip>> (<<Ontologically-controlled data management of laboratory and computational experiments in chemistry and materials science\unskip>>)

    Программа выполняется в рамках темы выпускной квалификационной работы в соответствии с учебным планом подготовки бакалавров по направлению 09.03.04 «Программная инженерия» Национального исследовательского университета «Высшая школа экономики», факультет компьютерных наук.
    \newpage


    \section{Назначение разработки}

    \subsection{Функциональное назначение}
    Разрабатываемое решение предоставляет возможность создавать, описывать, просматривать, редактировать и экспортировать эксперименты лабораторных и вычислительных экспериментов. Эксперименты описываются с привязкой к онтологиям из расширяемого набора возможных онтологий предметной области.

    \subsection{Эксплуатационное назначение}
    Приложение должно использоваться в лабораториях химии и материаловедения, в очередь РТУ МИРЭА. Конечными пользователями являются группы научных сотрудников, студенты, а также ML-разработчики и аналитики данных.
    \newpage


    \section{Требования к программе}

    \subsection{Функциональные требования}

    \subsubsection{Требования к работе с лабораторными экспериментами} \label{label:lab_exp}
    \begin{enumerate}
        \item Возможность создать эксперимент с нуля.
        \item Возможность импортировать произвольный эксперимент в подготовленном формате, совместимом с БД.
        \item Возможность присвоить название и описание как созданному с нуля, так и импортированному эксперименту.
    \end{enumerate}

    \subsubsection{Требования к работе с вычислительными экспериментами} \label{label:comp_exp}
    \begin{enumerate}
        \item Возможность создать шаблон эксперимента.
        \item Возможность сохранить результат отдельного запуска эксперимента вместе с использованным входом и параметрами эксперимента и дополнительным произвольным контекстом.
        \item Возможность визуализировать результаты отдельного запуска эксперимента.
    \end{enumerate}

    \subsubsection{Требования к базе данных}
    \begin{enumerate}
        \item Возможность динамического изменения схемы данных без значительных затрат времени на изменение структуры всей базы данных.
        \item Возможность подключения и обмена данными с другими приложениями и сервисами через API.
        \item Поддержка транзакций для обеспечения целостности данных при выполнении операций добавления, обновления и удаления.
    \end{enumerate}

    \subsubsection{Требования к серверной части} \label{"lable:backend"}
    Система должна предоставлять REST API \cite{technologies:REST} интерфейс для взаимодействия с экспериментами.
    \begin{enumerate}
        \item Предоставлять возможность аутентификации в системе с использованием RBAC \cite{arch:RBAC}.
        \item Возвращать список экспериментов конкретного пользователя.
        \item Создание эксперимента для конкретного пользователя.
        \item Экспорт данных эксперимента в популярные форматы: CSV\cite{Format:CSV}, PDF.
    \end{enumerate}

    \subsection{Требования к интерфейсу}

    \subsubsection{Требования к интерфейсу пользователя}
    Фронтенд должен иметь интуитивно понятный веб-интерфейс и предоставлять доступ ко всем функциям, описанным в параграфах \ref{label:lab_exp}, \ref{label:comp_exp}, \ref{"lable:backend"}.

    \subsection{Требования к надежности}
    \begin{enumerate}
        \item Сохранение работоспособности и обеспечение восстановления своих функций при возникновении некорректных API запросов.
        \item Логирование всех внештатных ситуаций для возможности анализа сбоев.
        \item Защита программы от несанкционированного доступа за счет использования идентификации пользователя и сессионной модели.
        \item Защита паролей пользователей за счет сохранения в базе и пересылки криптографических хэшей.
    \end{enumerate}

    \subsection{Требования к формату входных и выходных данных}
    На вход приложение получает либо запрос на создание, просмотр или импорт/экспорт эксперимента. Запрос валидируется и в случае успеха возвращается форма создания нового эксперимента, существующий эксперимент в формате приложения или одном из возможных для экспорта формате, в зависимости от типа запроса.

    \subsection{Условия эксплуатации}

    \subsubsection{Климатические условия}
    Климатические условия должны совпадать с климатическими условиями эксплуатации устройства пользователя или физического расположения сервера.

    \subsubsection{Требования к пользователю}
    Пользователь должен быть ознакомлен с документами <<Руководство программиста  <<Веб-приложение для ведения лабораторных и вычислительных экспериментов с поддержкой онтологий\unskip>> и <<Руководство пользователя <<Веб-приложение для ведения лабораторных и вычислительных экспериментов с поддержкой онтологий\unskip>>, а также разбираться в терминологии (см. Приложение \ref{terms}).

    \subsection{Требования к составу и параметру технических средств}
    Для надежной работы серверной части проекта требуется выделенное серверное оборудование с доступом к сети Интернет. Сервер может быть создан как в виде отдельного физического устройства, так и как виртуальный сервер.

    Для стабильного функционирования серверной части к серверу предъявляются следующие требования:
    \begin{enumerate}
        \item Процессор – не менее одного процессора со спецификацией: тактовая частота 2,4 ГГц, 2 ядра
        \item Оперативная память – не менее 8 ГБ
        \item Жесткий диск или твердотельный накопитель – не менее 16 ГБ
        \item Сетевой адаптер – не менее 1 порта Fast Ethernet
    \end{enumerate}

    Для стабильного функционирования клиентской части к стационарным/портативным компьютерам предъявляются следующие требования:
    \begin{enumerate}
        \item Операционная система:
        \begin{enumerate}
            \item[a.] Windows 8 или новее;
            \item[b.] macOS 10.12 или новее;
            \item[c.] Linux: любое современное дистрибутивное ядро с поддержкой GUI.
        \end{enumerate}
        \item Процессор – не менее одного процессора со спецификацией: тактовая частота 2,0 ГГц, 2 ядра
        \item Оперативная память – не менее 4 ГБ;
        \item Жесткий диск или твердотельный накопитель – не менее 100 МБ свободного места для временных файлов
        \item Поддержка современных браузеров с поддержкой HTML5 и JavaScript
        \item Доступ к сети Интернет
    \end{enumerate}

    \subsection{Требования к информационной и программной совместимости}

    \subsubsection{Требования к исходным кодам и языкам программирования}
    Исходные коды клиентской части должны быть написаны на языке JavaScript.
    Исходные коды серверной части должны быть написаны на языке Python с помощью фреймворка FastAPI\cite{Framework:FastAPI}.

    \subsubsection{Требования к программным средствам, используемым программой}
    Для эксплуатации программного продукта необходимо наличие следующих компонентов:
    \begin{enumerate}
        \item Docker;
        \item Браузеры: Google Chrome, Yandex Browser. Веб-браузер должен поддерживать HTML5, включена поддержка JavaScript версии EcmaScript6 и cookie файлов
    \end{enumerate}

    \subsection{Требования к маркировке и упаковке}
    Приложение должно быть доступно для установки из архива проекта, при скачивании из системы LMS НИУ ВШЭ.

    \subsection{Требования к транспортированию и хранению}
    Программное изделие может храниться и транспортироваться на облачном хранилище.
    \newpage


    \section{Требования к программной документации}
    Состав программной документации должен включать в себя следующие компоненты:
    \begin{enumerate}
        \item Техническое задание <<Веб-приложение для ведения лабораторных и вычислительных экспериментов с поддержкой онтологий\unskip>> (ГОСТ 19.201-78) \label{tz}
        \item Программа и методика испытаний <<Веб-приложение для ведения лабораторных и вычислительных экспериментов с поддержкой онтологий\unskip>> (ГОСТ 19.301-78) \label{pmi}
        \item Руководство оператора <<Веб-приложение для ведения лабораторных и вычислительных экспериментов с поддержкой онтологий\unskip>> (ГОСТ 19.505-79) \label{ro}
        \item Текст программы <<Веб-приложение для ведения лабораторных и вычислительных экспериментов с поддержкой онтологий\unskip>> (ГОСТ 19.401-78) \label{tp}
    \end{enumerate}

    \indent
    Вся документация должна быть составлена согласно ЕСПД (ГОСТ 19.101-77, 19.104-78, 19.105-78, 19.106-78 и ГОСТ к соответствующим документам (см. выше)) \cite{TZ:gost0, TZ:gost1, TZ:gost2, TZ:gost3, TZ:gost4, TZ:gost5, TZ:gost6, TZ:gost7, TZ:gost8, TZ:gost9}. Все документы сдаются в электронном виде в составе выпускной квалификационной работы LMS НИУ ВШЭ.
    \newpage


    \section{Технико-экономические показатели}

    \subsection{Ориентировочная экономическая эффективность}
    В рамках данного проекта экономическая эффективность не предусмотрена.

    \subsection{Предполагаемая потребность}
    Данный программный продукт будет интересен в первую очередь научным сотрудникам, работающим в лабораториях, проводящих лабораторные и вычислительные эксперименты, а также студентам, принимающим участие в работе таких лабораторий. Кроме того, им смогут пользоваться аналитики данных и ML-разработчики для экспорта данных с целью дальнейшего использования при построении моделей машинного обучения.

    \subsection{Экономические преимущества разработки по сравнению с лучшими отечественными и зарубежными образцами или аналогами}
    \begin{table}[h!]
        \begin{tabular}{|p{3cm}|p{1.5cm}|p{2.5cm}|p{2cm}|p{2.5cm}|p{2.5cm}|}
            \hline
            Решение                          & Год выпуска & Коммерческое ПО & Поддержка онтологий & Порог входа         & Импорт и экспорт данных \\
            \hline
            SciNote                          & 2015        & +               & Очень ограничена    & Умеренно высокий    & Есть                    \\
            \hline
            LabGuru                          & 2011        & +/--            & Очень ограничена    & Относительно низкий & Есть                    \\
            \hline
            CellPort                         & 2021        & +               & Нет                 & Высокий             & Есть                    \\
            \hline
            GeneMod                          & 2018        & +               & Очень ограничена    & Умеренно высокий    & Есть                    \\
            \hline
            Веб-приложение для ведения лабораторных и вычислительных экспериментов с поддержкой онтологий\unskip & 2025        & --              & Да                  & Низкий              & Есть                    \\
            \hline
        \end{tabular}
        \caption{Сравнение с наиболее близкими решениями}
        \label{tab:Табл. 1}
    \end{table}

    \newpage


    \section{Стадии и этапы разработки}

    \subsection{Техническое задание}

    \subsubsection{Подготовительные работы}
    \begin{enumerate}
        \item Постановка задачи.
        \item Сбор исходных теоретических материалов.
        \item Определение структуры входных и выходных данных.
        \item Предварительный выбор методов решения задач.
    \end{enumerate}

    \subsubsection{Разработка и утверждение технического задания}
    \begin{enumerate}
        \item Определение требований к программе.
        \item Определение требований к техническим средствам.
        \item Определение стадий, этапов и сроков разработки программы и документации на неё.
        \item Выбор используемых технологий.
        \item Согласование и утверждение технического задания.
    \end{enumerate}

    \subsection{Рабочий проект}

    \subsubsection{Разработка программы}
    \begin{enumerate}
        \item Разработка схемы базы данных.
        \item Разработка архитектуры программы.
        \item Программирование и отладка программы.
    \end{enumerate}

    \subsubsection{Разработка программной документации}
    \begin{enumerate}
        \item Разработка программных документов в соответствии с требованиями ГОСТ 19.101-77 \cite{TZ:gost0}.
    \end{enumerate}

    \subsubsection{Испытания программы}
    \begin{enumerate}
        \item Разработка, согласование и утверждение порядка и методики испытаний.
        \item Корректировка программы и программной документации по результатам испытаний.
    \end{enumerate}

    \subsection{Внедрение}

    \subsubsection{Подготовка и передача программы.}
    \begin{enumerate}
        \item Подготовка программы и программной документации для презентации и защиты.
        \item Представление разработанного программного продукта руководителю и получение отзыва.
        \item Загрузка текста ВКР в систему Антиплагиат через ЛМС НИУ ВШЭ.
        \item Загрузка материалов выпускной квалификационной работы в ЛМС НИУ ВШЭ.
        \item Защита программного продукта комиссии.
    \end{enumerate}
    \newpage


    \section{Порядок контроля и приемки}

    \subsection{Виды испытаний}
    Проверка программного продукта, в том числе и на соответствие техническому заданию, осуществляется исполнителем вместе с заказчиком согласно «Программе и методике испытаний», а также пункту 5.2.

    \subsection{Общие требования к приемке работы}
    Защита выполненного проекта осуществляется комиссии, состоящей из преподавателей департамента программной инженерии, в утверждённые приказом декана ФКН сроки.

    Контроль и приемка разработки осуществляются в соответствии с документом <<Программа и методика испытаний <<Веб-приложение для ведения лабораторных и вычислительных экспериментов с поддержкой онтологий\unskip>>.
    \newpage

    \addition{Используемые понятия и определения}{terms}
    \phantomsection
\addcontentsline{toc}{section}{Основные определения, термины и сокращения}

\begin{description}[style=unboxed,leftmargin=0cm]

    \item[\textbf{Валидность}] -- степень соответствия полученных данных реальности.

    \item[\textbf{Вычислительный эксперимент}] -- метод исследования, при котором математические модели и компьютерные расчёты используются для анализа сложных систем и прогнозирования их поведения.

    \item[\textbf{Домен (предметная область)}] -- область знаний, к которой относится конкретное программное обеспечение.

    \item[\textbf{Интероперабельность}] -- возможность обмена и понимания данных между разными системами на основе их смыслового значения.

    \item[\textbf{Клиент-серверная архитектура}] -- принцип проектирования ПО, согласно которому клиент запрашивает данные у сервера, который их обрабатывает и отправляет обратно.

    \item[\textbf{Лабораторный эксперимент}] -- метод научного исследования, проводимый в контролируемых условиях лаборатории для проверки гипотез, изучения свойств объектов или выявления закономерностей.

    \item[\textbf{Онтология}] -- формализованное представление знаний в определённой предметной области, включающее определения понятий, их свойства и взаимосвязи между ними. Используется в искусственном интеллекте, базах знаний и прочих областях.

    \item[\textbf{Слоистая архитектура}] -- принцип проектирования ПО, согласно которому система разделена на слои, каждый из которых выполняет свою роль.

    \item[\textbf{СУБД (Система управления базами данных)}] -- программное обеспечение для создания, управления и работы с базами данных.

    \item[\textbf{API (Application Programming Interface)}] -- интерфейс программирования приложений, набор правил и инструментов для взаимодействия между программными компонентами.

    \item[\textbf{ELN (Electronic Laboratory Notebook)}] -- электронный лабораторный журнал, который используется для ведения, хранения и управления научными записями, заменяя традиционные бумажные записи.

    \item[\textbf{JWT (JSON Web Token)}] -- компактный и безопасный способ передачи информации между сторонами в виде JSON-объекта, часто используется для аутентификации.

    \item[\textbf{ORM (Object-Relational Mapping)}] -- технология, позволяющая работать с базами данных с использованием объектно-ориентированных подходов вместо SQL-запросов.

    \item[\textbf{SPA (Single Page Application)}] -- веб-приложение, которое загружается один раз и обновляет контент без полной перезагрузки страницы, улучшая пользовательский опыт.

\end{description}
    \newpage

    \printbibliography[title=Список источников, heading=bibintoc]

    \newpage

    \listRegistration

\end{document} % конец документа