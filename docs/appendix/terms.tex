\phantomsection
\addcontentsline{toc}{section}{Основные определения, термины и сокращения}

\begin{description}[style=unboxed,leftmargin=0cm]

    \item[\textbf{Валидность}] – степень соответствия полученных данных реальности.

    \item[\textbf{Вычислительный эксперимент}] -- метод исследования, при котором математические модели и компьютерные расчёты используются для анализа сложных систем и прогнозирования их поведения.

    \item[\textbf{Домен (предметная область)}] -- область знаний, к которой относится конкретное программное обеспечение.

    \item[\textbf{Интероперабельность}] – возможность обмена и понимания данных между разными системами на основе их смыслового значения.

    \item[\textbf{Клиент-серверная архитектура}] -- принцип проектирования ПО, согласно которому клиент запрашивает данные у сервера, который их обрабатывает и отправляет обратно.

    \item[\textbf{Лабораторный эксперимент}] -- метод научного исследования, проводимый в контролируемых условиях лаборатории для проверки гипотез, изучения свойств объектов или выявления закономерностей.

    \item[\textbf{Онтология}] -- формализованное представление знаний в определённой предметной области, включающее определения понятий, их свойства и взаимосвязи между ними. Используется в искусственном интеллекте, базах знаний и семантическом вебе.

    \item[\textbf{Слоистая архитектура}] -- принцип проектирования ПО, согласно которому система разделена на слои, каждый из которых выполняет свою роль.

    \item[\textbf{СУБД (Система управления базами данных)}] -- программное обеспечение для создания, управления и работы с базами данных.

    \item[\textbf{API (Application Programming Interface)}] -- интерфейс программирования приложений, набор правил и инструментов для взаимодействия между программными компонентами.

    \item[\textbf{ELN (Electronic Laboratory Notebook)}] -- электронный лабораторный журнал, который используется для ведения, хранения и управления научными записями, заменяя традиционные бумажные записи.

    \item[\textbf{JWT (JSON Web Token)}] -- компактный и безопасный способ передачи информации между сторонами в виде JSON-объекта, часто используется для аутентификации.

    \item[\textbf{ORM (Object-Relational Mapping)}] -- технология, позволяющая работать с базами данных с использованием объектно-ориентированных подходов вместо SQL-запросов.

    \item[\textbf{SPA (Single Page Application)}] -- веб-приложение, которое загружается один раз и обновляет контент без полной перезагрузки страницы, улучшая пользовательский опыт.

\end{description}