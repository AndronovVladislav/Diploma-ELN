\documentclass[a4paper,12pt]{article}
\usepackage{styledoc19}

\programmingLanguage{Scala}
\lstset{
    basicstyle=\ttfamily,
    breaklines=true,
}

\usepackage[T1]{fontenc}
\catcode`\_=12

\renewcommand{\numberline}[1]{#1~}

\begin{document} % конец преамбулы, начало документа

    % \year{2022}
    \docNumber{RU.17701729.04.01-01 12 01-1}
    \docFormat{Текст программы}
    \student{БПИ 183}{А. А. Глушко}

    \project{Автоматическая генерация и проверка учебных заданий на основе онтологии предметной области для систем персонализированного адаптивного обучения}

    \supervisor{Доцент Факультет компьютерных \vfill наук анализа данных и \vfill искусственного интеллекта  \vfill }
    {А. А. Незнанов}

    \firstPage
    \newpage
    \secondPage
    \newpage
    \thirdPageWithoutTOC
    \section*{Аннотация}

    В данном документе представлена информация о репозитории, в котором хранится исходный код программы <<Онтологически-контролируемое управление данными лабораторных и вычислительных экспериментов в задачах химии и материаловедения\unskip>>. Данное приложение является частью системы <<Интерактивный тренажёр с онтологически-контролируемой генерацией и проверкой учебных заданий по физике>> и разрабатывалось в общем проекте на GitHub, который находится по адресу \url{https://github.com/Badcat330/EdMachine}.

    \newpage


    \section{Директория interaction-backend}
    В данной директории расположен исходный код серверной части расчетной подсистемы и вспомогательные файлы для сборки докер образа сервиса.

    \subsection{Директория src}
    В данной директории располагается исходный код сервиса, а также скрипт для инициализации сервиса и конфигурационные файлы.

    \subsubsection{Директория controllers}
    В данной директории расположен исходный код контроллеров, обрабатывающих запросы к серверу. В директории расположены следующие файлы:
    \begin{itemize}
        \item assessment.py: контроллер, обрабатывающий запросы к контрольно-измерительным материалам;
        \item health\_check\_controller.py: контроллер, обрабатывающий запрос на проверку состояния сервиса;
        \item notification.py: контроллер, обрабатывающий запросы на отправку email уведомлений;
        \item type.py: контроллер, обрабатывающий запросы на доступ к данным из онтологии учебного процесса;
        \item user.py: контроллер, обрабатывающий запросы к данным о пользователях.
    \end{itemize}

    \subsubsection{Модуль model}
    Модуль содержит описание ORM модели с помощью SQL Alchemy.

    \subsubsection{Модуль scafolding}
    В данном модуле реализован алгоритм проверки корректности выполнения задания пользователем и побора релевантных материалов в случае некорректного ответа.

    \subsubsection{Директория services}
    В данной директории расположена реализация следующих сервисов:
    \begin{itemize}
        \item email\_client.py: сервис для отправки писем;
        \item materials.py: сервис для взаимодействия с подсистемой учебных материалов;
        \item ontology.py: сервис для взаимодействия с подсистемой управления онтологиями;
        \item task\_bank.py: сервис для взаимодействия с банком контрольно-измерительных материалов.
    \end{itemize}

    \subsubsection{Модуль utils}
    В данном модуле реализован набор инструментов для проверки входных данных запросов.


    \section{Директория database}
    В данном модуле находится скрипт для создания базы данных. Он может быть использован для создания тестовой базы данных и проведения тестирования.


    \section{Директория task-generator}
    В данной директории расположена директория с исходным кодом генератора контрольно-измерительных материалов и вспомогательные файлы для сборки докер образа генератора.

    \subsection{Директория src}
    В данной директории расположен исходный код генератора контрольно-измерительных материалов, а также скрипт для инициализации сервиса и конфигурационные файлы.

    \subsubsection{Директория controllers}
    В данной директории расположен исходный код контроллеров, обрабатывающих запросы к серверу. В директории расположены следующие файлы:
    \begin{itemize}
        \item generator\_controller.py: контроллер, обрабатывающий запрос на генерацию контрольно-измерительного материала;
        \item health\_check\_controller.py: контроллер, обрабатывающий запрос на проверку состояния сервиса.
    \end{itemize}

    \subsubsection{Директория schemas}
    В данной директории расположены JSON\cite{Format:JSON} Schema, описывающие модель данных контрольно-измерительного материала.

    \subsubsection{Директория services}
    В данной директории расположен сервис для взаимодействия с подсистемой управления онтологиями.

    \subsubsection{Модуль solver}
    В данном модуле описан алгоритм получения аналитического решения контрольно-измерительного материала по его модели и вычисления ответов на вопросы на основе полученного аналитического решения.

    \subsubsection{Модуль text_generation}
    В данном модуле реализован алгоритм генерации текста контрольно-измерительного материала по его шаблону.

    \subsubsection{Модуль utils}
    В данном модуле реализован набор инструментов для удобной работы с моделью задачи.


    \section{Директория tutor-user-interaction}
    В данной директории расположен исходный код клиентской части расчетной подсистемы, а также конфигурационные файлы для сборки докер образа сервиса и функционирования автоматических форматеров.

    \subsection{Директория components}
    В данной директории расположены React компоненты, использованные в ходе реализации пользовательского интерфейса.

    \subsection{Директория pages}
    В данной директории расположен исходный код страниц Web приложения.

    \subsection{Директория public}
    В данной директории расположены вспомогательные медиа файлы, такие как иконки, картинки для фона и так далее.

    \subsection{Директория styles}
    В данной директории расположены стили для оформления страниц пользовательского интерфейса.


    \newpage
    \listRegistration
\end{document}