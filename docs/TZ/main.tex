\documentclass[a4paper,12pt,reqno]{article}

\usepackage{styledoc19}
\addbibresource{references.bib}

\begin{document}

    \docNumber{RU.17701729.12.20-01 ТЗ 01-1}
    \docFormat{Техническое задание}
    \student{БПИ 213}{В. А. Андронов}
    \project{Веб-приложение для ведения лабораторных и вычислительных экспериментов с поддержкой онтологий\unskip}
    \supervisor{Доцент департамента \vfill программной инженерии \vfill факультета компьютерных \vfill наук, кандидат \vfill технических наук}{О. В. Максименкова}

    \firstPage
    \newpage

    \secondPage
    \newpage

    \thirdPage
    \newpage


    \section{Введение}

    \subsection{Наименование программы}

    \subsubsection{Наименование программы на русском языке}
    Веб-приложение для ведения лабораторных и вычислительных экспериментов с поддержкой онтологий\unskip

    \subsubsection{Наименование программы на английском языке}
    Ontologically-controlled data management of laboratory and computational experiments in chemistry and materials science\unskip

    \subsection{Краткая характеристика области применения}
    Онтологически-контролируемое управление данными представляет собой современный подход к систематизации и обработке информации в химии и материаловедении. Данный метод основывается на использовании онтологий — формальных описаний концепций и их взаимосвязей в конкретной предметной области, что позволяет создать единое семантическое пространство для обмена знаниями и данными~\cite{ontology:base}.

    В условиях увеличивающихся объемов данных и сложных междисциплинарных исследований, необходимость в эффективных инструментах управления информацией становится все более очевидной. Онтологически-контролируемое управление данными позволяет не только организовывать большие объемы информации, но и обеспечивать их интероперабельность. Это упрощает процессы интеграции данных из различных источников и способствует более тесному сотрудничеству между научными коллективами и организациями, занимающимися исследованиями в смежных областях.
    \newpage


    \section{Основания для разработки}

    \subsection{Документы, на основании которых ведется разработка}
    Приказ декана факультета компьютерных наук И.В. Аржанцева <<Об утверждении тем, руководителей выпускных квалификационных работ студентов образовательной программы «Программная инженерия» факультета компьютерных наук>> № 2.3-02/040425-1 от 04.04.2025.

    \subsection{Наименование темы разработки}
    Наименование темы разработки – <<Веб-приложение для ведения лабораторных и вычислительных экспериментов с поддержкой онтологий\unskip>> (<<Ontologically-controlled data management of laboratory and computational experiments in chemistry and materials science\unskip>>)

    Программа выполняется в рамках темы выпускной квалификационной работы в соответствии с учебным планом подготовки бакалавров по направлению 09.03.04 «Программная инженерия» Национального исследовательского университета «Высшая школа экономики», факультет компьютерных наук.
    \newpage


    \section{Назначение разработки}

    \subsection{Функциональное назначение}
    Разрабатываемое решение предоставляет возможность создавать, описывать, просматривать, редактировать и экспортировать эксперименты лабораторных и вычислительных экспериментов. Эксперименты описываются с привязкой к онтологиям из расширяемого набора возможных онтологий предметной области.

    \subsection{Эксплуатационное назначение}
    Приложение может использоваться в научных учреждениях, занимающихся в первую очередь химией и материаловедением, хотя возможна доработка для любой области, в которой уместно понятие эксперимента и для которой существуют подходящие онтологии. Конечными пользователями являются группы научных сотрудников, студенты, а также ML-разработчики и аналитики данных.
    \newpage


    \section{Требования к программе}

    \subsection{Функциональные требования} \label{funcreq}

    \begin{enumerate}
        \item Управление пользователями

        \begin{enumerate}[label=\arabic{enumi}.\arabic*.]
            \item Пользователь должен иметь возможность зарегистрироваться в системе с указанием логина и пароля.
            \item Пользователь должен иметь возможность войти в систему, используя свои учетные данные (логин и пароль) и получать access-токен и refresh-токен.
            \item Для получения нового access-токена пользователь может использовать эндпоинт, предоставив действующий refresh-токен.
        \end{enumerate}

        \item Управление экспериментами

        \begin{enumerate}[label=\arabic{enumi}.\arabic*.]
            \item Пользователь должен иметь возможность создавать лабораторные эксперименты, передавая описание эксперимента и задавая путь к нему.
            \item Пользователь должен иметь возможность создавать вычислительные эксперименты с обязательной привязкой к шаблону, передавая путь и идентификатор шаблона.
            \item Пользователь должен иметь возможность получать список своих экспериментов с фильтрацией по ключам.
            \item Пользователь должен иметь возможность обновлять данные лабораторных экспериментов, передавая обновленные данные.
            \item Пользователь должен иметь возможность обновлять данные вычислительных экспериментов.
            \item Пользователь должен иметь возможность получать полную информацию об эксперименте любого типа по его идентификатору.
            \item Пользователь должен иметь возможность удалять эксперимент любого типа.
        \end{enumerate}

        \item Интеграция с онтологиями

        \begin{enumerate}[label=\arabic{enumi}.\arabic*.]
            \item Пользователь должен иметь возможность получать список доступных онтологий.
            \item Пользователь должен иметь возможность получать подробности об онтологии, с ограничением по количеству элементов.
            \item Пользователь должен иметь возможность привязать столбцы таблицы эксперимента к онтологиям.
        \end{enumerate}

        \item Управление шаблонами вычислительных экспериментов

        \begin{enumerate}[label=\arabic{enumi}.\arabic*.]
            \item Пользователь должен иметь возможность создавать шаблоны вычислительных экспериментов, передавая путь и данные шаблона.
            \item Пользователь должен иметь возможность обновлять шаблоны.
            \item Пользователь должен иметь возможность получать список всех доступных шаблонов.
            \item Пользователь должен иметь возможность удалять шаблоны.
        \end{enumerate}

        \item Экспорт, интеграция и API

        \begin{enumerate}[label=\arabic{enumi}.\arabic*.]
            \item Пользователь должен иметь возможность импортировать эксперимент любого типа, передавая JSON-файл в следующем формате:
            \begin{lstlisting}[frame=single, basicstyle=\footnotesize\ttfamily, label={lst:lab_import}, caption={Формат данных для импорта лабораторого эксперимента},captionpos=b, breaklines=true, breakatwhitespace=true]
{
  "path": string,
  "description": string,
  "columns": [
    {
      "name": string,
      "ontology": string,
      "ontology_ref": string,
      "is_main": boolean
    }
  ],
  "measurements": [
    {
      "row": int,
      "column": int,
      "value": string
    }
  ]
}
            \end{lstlisting}
            \begin{lstlisting}[frame=single, basicstyle=\footnotesize\ttfamily, label={lst:comp_import}, caption={Формат данных для импорта вычислительного эксперимента},captionpos=b, breaklines=true, breakatwhitespace=true]
{
  "path": string,
  "template": {
    "path": string,
    "description": string,
    "input": {  // An object describing the input parameters. The values are links to ontologies in the "ontology:uri" format.
      "param_1": string,
      "param_2": string
    },
    "output": {
      "param_1": string,
      "param_2": string
    },
    "parameters": {
      "param_1": string,
      "param_2": string
    },
    "context": {
      "not_compulsory_field_0": string,
      "not_compulsory_field_1": string
    }
  },
  "data": [  // A list of objects representing a row (one run) of a computational experiment
    {
      "input": {
        "param_1": string,  // The actual value at launch
        "param_2": string
      },
      "output": {
        "param_1": string,
        "param_2": string
      },
      "parameters": {
        "param_1": string,
        "param_2": string
      },
    }
  ]
}
            \end{lstlisting}
            \item Поддержка REST~\cite{arch:REST} API для взаимодействия с внешними системами.
        \end{enumerate}
    \end{enumerate}

    \subsection{Требования к интерфейсу}

    \subsubsection{Требования к интерфейсу пользователя}\label{ui-requirements}

    \begin{enumerate}
        \item \textbf{Основные страницы интерфейса}
        \begin{enumerate}[label=\arabic{enumi}.\arabic*.]
            \item Веб-приложение должно состоять из следующих основных страниц:
            \begin{enumerate}[label=\arabic{enumi}.\arabic{enumii}.\arabic*.]
                \item \textbf{Страница входа}: страница, на которой пользователь вводит свои учетные данные (логин и пароль) для аутентификации. Страница должна включать форму для ввода данных и кнопку для отправки формы.
                \item \textbf{Страница регистрации}: страница для создания нового аккаунта с формой для ввода логина, пароля и подтверждения пароля.
                \item \textbf{Страница экспериментов}: страница, на которой пользователи могут просматривать, создавать, редактировать и удалять эксперименты. Страница должна содержать список экспериментов и кнопки для создания нового эксперимента, а также фильтры для поиска.
                \item \textbf{Страница шаблонов вычислительных экспериментов}: страница, на которой пользователи могут просматривать, создавать, редактировать и удалять шаблоны для вычислительных экспериментов.
                \item \textbf{Страница онтологий}: страница для отображения списка доступных онтологий и их подробностей. Пользователь должен иметь возможность привязать столбцы эксперимента к выбранной онтологии.
                \item \textbf{Страница настроек пользователя}: страница для изменения пароля и управления личными данными пользователя.
            \end{enumerate}
        \end{enumerate}

        \item \textbf{Навигация и взаимодействие с элементами}
        \begin{enumerate}[label=\arabic{enumi}.\arabic*.]
            \item Приложение должно обеспечивать интуитивно понятную навигацию с использованием меню, которое включает:
            \begin{enumerate}[label=\arabic{enumi}.\arabic{enumii}.\arabic*.]
                \item Ссылки на страницы экспериментов, шаблонов и онтологий.
                \item Кнопки для выхода из системы и перехода в настройки.
                \item Фильтры и поля поиска для быстрого доступа к нужным экспериментам или шаблонам.
            \end{enumerate}
            \item Каждая страница должна содержать ясные и понятные кнопки для выполнения действий, таких как:
            \begin{enumerate}[label=\arabic{enumi}.\arabic{enumii}.\arabic*.]
                \item Кнопка для создания нового эксперимента, шаблона или загрузки данных.
                \item Кнопка для удаления эксперимента или шаблона с подтверждением удаления.
                \item Кнопка для сохранения изменений в эксперименте или шаблоне.
                \item Кнопка для загрузки файла (например, для импорта/экспорта данных в формате JSON).
            \end{enumerate}
        \end{enumerate}

        \item \textbf{Форма ввода данных}
        \begin{enumerate}[label=\arabic{enumi}.\arabic*.]
            \item На странице создания/редактирования эксперимента должна быть форма для ввода:
            \begin{enumerate}[label=\arabic{enumi}.\arabic{enumii}.\arabic*.]
                \item Названия эксперимента.
                \item Описания эксперимента.
                \item Выбора типа эксперимента (лабораторный или вычислительный).
                \item Загрузки данных эксперимента в формате JSON или CSV.
                \item Привязки столбцов к онтологиям.
            \end{enumerate}
            \item На странице регистрации должна быть форма для ввода:
            \begin{enumerate}[label=\arabic{enumi}.\arabic{enumii}.\arabic*.]
                \item Логина пользователя.
                \item Пароля пользователя.
                \item Подтверждения пароля.
            \end{enumerate}
            \item На странице входа должна быть форма для ввода:
            \begin{enumerate}[label=\arabic{enumi}.\arabic{enumii}.\arabic*.]
                \item Логина пользователя.
                \item Пароля пользователя.
            \end{enumerate}
        \end{enumerate}

        \item \textbf{Отображение и визуализация данных}
        \begin{enumerate}[label=\arabic{enumi}.\arabic*.]
            \item На странице экспериментов должны отображаться:
            \begin{enumerate}[label=\arabic{enumi}.\arabic{enumii}.\arabic*.]
                \item Список всех доступных экспериментов с кратким описанием и датой создания.
                \item Для каждого эксперимента должна быть возможность просмотра, редактирования и удаления.
                \item Фильтры для поиска экспериментов по названию, типу и дате.
                \item Кнопка для импорта и экспорта экспериментов в формате JSON или CSV.
            \end{enumerate}
            \item На странице онтологий должны отображаться:
            \begin{enumerate}[label=\arabic{enumi}.\arabic{enumii}.\arabic*.]
                \item Список доступных онтологий с возможностью просмотра подробной информации о каждой.
                \item Возможность привязки столбцов таблицы эксперимента к онтологиям.
            \end{enumerate}
            \item Для каждого эксперимента должна быть визуализация измерений (например, таблица с результатами).
        \end{enumerate}

        \item \textbf{Интерфейс для работы с API}
        \begin{enumerate}[label=\arabic{enumi}.\arabic*.]
            \item Приложение должно иметь интеграцию с REST API для работы с данными. Все запросы должны быть выполнены через стандартные HTTP методы (GET, POST, PUT, DELETE).
            \item Пользователь должен иметь возможность импортировать и экспортировать эксперименты, передавая данные в формате JSON.
            \item Вся информация должна быть передана и получена в формате JSON для обеспечения совместимости с внешними системами.
        \end{enumerate}

        \item \textbf{Уведомления и сообщения}
        \begin{enumerate}[label=\arabic{enumi}.\arabic*.]
            \item Приложение должно показывать уведомления о выполнении операций, таких как:
            \begin{enumerate}[label=\arabic{enumi}.\arabic{enumii}.\arabic*.]
                \item Успешное создание, обновление или удаление эксперимента.
                \item Ошибки при загрузке данных или сохранении изменений.
                \item Успешное выполнение импорта/экспорта данных.
            \end{enumerate}
            \item Все ошибки и предупреждения должны быть отображены в понятном виде с указанием причины и возможных решений.
        \end{enumerate}
    \end{enumerate}

    \subsection{Требования к надежности}
    \begin{enumerate}
        \item Сохранение работоспособности и обеспечение восстановления своих функций при возникновении некорректных API запросов.
        \item Логирование всех внештатных ситуаций для возможности анализа сбоев.
        \item Защита программы от несанкционированного доступа за счет использования идентификации пользователя и сессионной модели.
        \item Защита паролей пользователей за счет сохранения в базе и пересылки криптографических хэшей.
    \end{enumerate}

    \subsection{Требования к формату входных и выходных данных}
    На вход приложение получает либо запрос на создание, просмотр или импорт/экспорт эксперимента. Запрос валидируется и в случае успеха возвращается форма создания нового эксперимента, существующий эксперимент в формате приложения или одном из возможных для экспорта формате, в зависимости от типа запроса.

    \subsection{Условия эксплуатации}

    \subsubsection{Климатические условия}
    Климатические условия должны совпадать с климатическими условиями эксплуатации устройства пользователя или физического расположения сервера.

    \subsubsection{Требования к пользователю}
    Пользователь должен быть ознакомлен с документами <<Руководство программиста  <<Веб-приложение для ведения лабораторных и вычислительных экспериментов с поддержкой онтологий\unskip>> и <<Руководство пользователя <<Веб-приложение для ведения лабораторных и вычислительных экспериментов с поддержкой онтологий\unskip>>, а также разбираться в терминологии (см. Приложение~\ref{terms}).

    \subsection{Требования к составу и параметру технических средств}
    Для надежной работы серверной части проекта требуется выделенное серверное оборудование с доступом к сети Интернет. Сервер может быть создан как в виде отдельного физического устройства, так и как виртуальный сервер.

    Для стабильного функционирования серверной части к серверу предъявляются следующие требования:
    \begin{enumerate}
        \item Процессор – не менее одного процессора со спецификацией: тактовая частота 2,4 ГГц, 2 ядра
        \item Оперативная память – не менее 8 ГБ
        \item Жесткий диск или твердотельный накопитель – не менее 16 ГБ
        \item Сетевой адаптер – не менее 1 порта Fast Ethernet
    \end{enumerate}

    Для стабильного функционирования клиентской части к стационарным/портативным компьютерам предъявляются следующие требования:
    \begin{enumerate}
        \item Операционная система:
        \begin{enumerate}[label=\arabic{enumi}.\arabic*.]
            \item Windows 8 или новее;
            \item macOS 10.12 или новее;
            \item Linux: любое современное дистрибутивное ядро с поддержкой GUI;
        \end{enumerate}
        \item Процессор – не менее одного процессора со спецификацией: тактовая частота 2,0 ГГц, 2 ядра
        \item Оперативная память – не менее 4 ГБ;
        \item Жесткий диск или твердотельный накопитель – не менее 100 МБ свободного места для временных файлов
        \item Поддержка современных браузеров с поддержкой HTML5 и JavaScript
        \item Доступ к сети Интернет
    \end{enumerate}

    \subsection{Требования к информационной и программной совместимости}

    \subsubsection{Требования к исходным кодам и языкам программирования}
    Исходные коды клиентской части должны быть написаны на языке JavaScript.
    Исходные коды серверной части должны быть написаны на языке Python с помощью фреймворка FastAPI~\cite{Framework:FastAPI}.

    \subsubsection{Требования к программным средствам, используемым программой}
    Для эксплуатации программного продукта необходимо наличие следующих компонентов:
    \begin{enumerate}
        \item Docker;
        \item Браузеры: Google Chrome, Yandex Browser. Веб-браузер должен поддерживать HTML5, включена поддержка JavaScript версии EcmaScript6 и cookie файлов
    \end{enumerate}

    \subsection{Требования к маркировке и упаковке}
    Приложение должно быть доступно для установки из архива проекта, при скачивании из системы LMS НИУ ВШЭ.

    \subsection{Требования к транспортированию и хранению}
    Программное изделие может храниться и транспортироваться на облачном хранилище.
    \newpage


    \section{Требования к программной документации}
    Состав программной документации должен включать в себя следующие компоненты:
    \begin{enumerate}
        \item Техническое задание <<Веб-приложение для ведения лабораторных и вычислительных экспериментов с поддержкой онтологий\unskip>> (ГОСТ 19.201-78) \label{tz}
        \item Программа и методика испытаний <<Веб-приложение для ведения лабораторных и вычислительных экспериментов с поддержкой онтологий\unskip>> (ГОСТ 19.301-78) \label{pmi}
        \item Руководство оператора <<Веб-приложение для ведения лабораторных и вычислительных экспериментов с поддержкой онтологий\unskip>> (ГОСТ 19.505-79) \label{ro}
        \item Текст программы <<Веб-приложение для ведения лабораторных и вычислительных экспериментов с поддержкой онтологий\unskip>> (ГОСТ 19.401-78) \label{tp}
    \end{enumerate}

    \indent
    Вся документация должна быть составлена согласно ЕСПД (ГОСТ 19.101-77, 19.104-78, 19.105-78, 19.106-78 и ГОСТ к соответствующим документам (см. выше))~\cite{TZ:gost0, TZ:gost1, TZ:gost2, TZ:gost3, TZ:gost4, TZ:gost5, TZ:gost6, TZ:gost7, TZ:gost8, TZ:gost9, TZ:gost10, TZ:gost11}. Все документы сдаются в электронном виде в составе выпускной квалификационной работы LMS НИУ ВШЭ.
    \newpage


    \section{Технико-экономические показатели}

    \subsection{Ориентировочная экономическая эффективность}
    В рамках данного проекта экономическая эффективность не предусмотрена.

    \subsection{Предполагаемая потребность}
    Данный программный продукт будет интересен в первую очередь научным сотрудникам, работающим в лабораториях, проводящих лабораторные и вычислительные эксперименты, а также студентам, принимающим участие в работе таких лабораторий. Кроме того, им смогут пользоваться аналитики данных и ML-разработчики для экспорта данных с целью дальнейшего использования при построении моделей машинного обучения.

    \subsection{Экономические преимущества разработки по сравнению с лучшими отечественными и зарубежными образцами или аналогами}
    \begin{table}[h]
        \centering
        \resizebox{\textwidth}{!}{
            \begin{tabular}{|l|c|c|c|c|c|}
                \hline
                \textbf{Характеристика}                  & \textbf{OpenBIS}               & \textbf{LabArchives}          & \textbf{Benchling}                   & \textbf{Chemotion ELN}               & \textbf{Разрабатываемый сервис}             \\
                \hline
                Стоимость использования                  & \cellcolor{green!20}Бесплатно  & \cellcolor{red!20}Подписка    & \cellcolor{red!20}Подписка           & \cellcolor{green!20}Бесплатно        & \cellcolor{green!20}Бесплатно               \\
                \hline
                Поддержка онтологий                      & \cellcolor{red!20}--           & \cellcolor{red!20}--          & \cellcolor{red!20}--                & \cellcolor{red!20}--                & \cellcolor{green!20}+                      \\
                \hline
                Автоматическая проверка единиц измерения & \cellcolor{red!20}--           & \cellcolor{red!20}--         & \cellcolor{red!20}--                & \cellcolor{red!20}--                & \cellcolor{green!20}+                      \\
                \hline
                Экспорт/импорт данных                    & \cellcolor{yellow!20}CSV, JSON & \cellcolor{yellow!20}CSV, PDF & \cellcolor{green!20}CSV, JSON, PDF   & \cellcolor{yellow!20}CSV, JSON       & \cellcolor{green!20}CSV, JSON, PDF          \\
                \hline
                Поддержка REST API                       & \cellcolor{green!20}+          & \cellcolor{green!20}+         & \cellcolor{green!20}+               & \cellcolor{green!20}+               & \cellcolor{green!20}+                      \\
                \hline
                Разграничение ролей                      & \cellcolor{green!20}+          & \cellcolor{green!20}+         & \cellcolor{green!20}+               & \cellcolor{green!20}+               & \cellcolor{green!20}+                      \\
                \hline
                Хранение онтологий                       & \cellcolor{red!20}--           & \cellcolor{red!20}--          & \cellcolor{red!20}--                & \cellcolor{red!20}--                & \cellcolor{green!20}+              \\
                \hline
                База данных                              & \cellcolor{green!20}PostgreSQL & \cellcolor{red!20}--          & \cellcolor{green!20}PostgreSQL       & \cellcolor{red!20}--                 & \cellcolor{green!20}PostgreSQL + Neo4j      \\
                \hline
                Основной стек технологий                 & \cellcolor{red!20}Java         & \cellcolor{red!20}--          & \cellcolor{red!20}--                 & \cellcolor{green!20}JavaScript, Ruby & \cellcolor{green!20}FastAPI, Vue.js         \\
                \hline
            \end{tabular}
        }
        \caption{Сравнение существующих ELN-решений и разрабатываемого сервиса}
        \label{tab:comparison}
    \end{table}

    \newpage


    \section{Стадии и этапы разработки}

    \begin{table}[h]
        \centering
        \resizebox{\textwidth}{!}{
            \begin{tabular}{|l|l|l|l|}
                \hline
                \makecell[lt]{Стадия \\разработки} & \makecell[lt]{Этапы \\разработки} & \makecell[lt]{Содержание работ} & \makecell[lt]{Сроки \\выполнения} \\
                \hline
                \multirow[t]{2}{*}{\makecell[lt]{Техническое \\задание}} & \multirow[t]{4}{*}{\makecell[lt]{Подготовительные \\работы}} & \makecell[lt]{Постановка задачи} & \makecell[lt]{01.11.2024- \\05.11.2024} \\
                \cline{3-4}
                & & \makecell[lt]{Сбор исходных теоретических \\материалов} & \makecell[lt]{05.11.2024- \\10.11.2024} \\
                \cline{3-4}
                & & \makecell[lt]{Определение структуры входных \\и выходных данных} & \makecell[lt]{10.11.2024- \\15.11.2024} \\
                \cline{3-4}
                & & \makecell[lt]{Предварительный выбор методов \\решения задач} & \makecell[lt]{15.11.2024- \\20.11.2024} \\
                \cline{2-4}

                & \multirow[t]{3}{*}{\makecell[lt]{Разработка и \\утверждение \\технического \\задания}} & \makecell[lt]{Определение требований к \\программе} & \makecell[lt]{20.11.2024- \\25.11.2024} \\
                \cline{3-4}
                & & \makecell[lt]{Определение требований к \\техническим средствам} & \makecell[lt]{25.11.2024- \\30.11.2024} \\
                \cline{3-4}
                & & \makecell[lt]{Определение стадий, этапов и сроков разработки \\программы и документации на неё} & \makecell[lt]{30.11.2024- \\05.12.2024} \\
                \cline{3-4}
                & & \makecell[lt]{Выбор используемых технологий} & \makecell[lt]{05.12.2024- \\10.12.2024} \\
                \cline{3-4}
                & & \makecell[lt]{Согласование и утверждение \\технического задания} & \makecell[lt]{10.12.2024- \\15.12.2024} \\
                \hline

                \multirow[t]{3}{*}{\makecell[lt]{Рабочий \\проект}} & \multirow[t]{3}{*}{\makecell[lt]{Разработка \\программы}} & \makecell[lt]{Разработка схемы базы \\данных} & \makecell[lt]{15.12.2024- \\25.12.2024} \\
                \cline{3-4}
                & & \makecell[lt]{Разработка архитектуры \\программы} & \makecell[lt]{25.12.2024- \\10.01.2025} \\
                \cline{3-4}
                & & \makecell[lt]{Программирование и отладка \\программы} & \makecell[lt]{10.01.2025- \\20.02.2025} \\
                \cline{2-4}
                & \makecell[lt]{Разработка \\программной \\документации} & \makecell[lt]{Разработка программных \\документов} & \makecell[lt]{20.02.2025- \\28.02.2025} \\
                \cline{2-4}
                & \multirow[t]{3}{*}{\makecell[lt]{Испытания \\программы}} & \makecell[lt]{Разработка, согласование и \\утверждение порядка и методики испытаний} & \makecell[lt]{01.03.2025- \\10.03.2025} \\
                \cline{3-4}
                & & \makecell[lt]{Проведение предварительных \\испытаний} & \makecell[lt]{10.03.2025- \\20.03.2025} \\
                \cline{3-4}
                & & \makecell[lt]{Корректировка программы и \\программной документации по \\результатам испытаний} & \makecell[lt]{20.03.2025- \\30.03.2025} \\
                \hline

                \multirow[t]{2}{*}{\makecell[lt]{Внедрение}} & \makecell[lt]{Подготовка и передача программы} & \makecell[lt]{Подготовка программы и программной документации \\для презентации и защиты} & \makecell[lt]{01.04.2025- \\05.04.2025} \\
                \cline{3-4}
                & & \makecell[lt]{Представление разработанного программного продукта \\руководителю и получение отзыва} & \makecell[lt]{05.04.2025- \\10.04.2025} \\
                \cline{3-4}
                & & \makecell[lt]{Загрузка текста ВКР в систему \\Антиплагиат через ЛМС НИУ ВШЭ} & \makecell[lt]{10.04.2025- \\15.04.2025} \\
                \cline{3-4}
                & & \makecell[lt]{Загрузка материалов выпускной квалификационной \\ работы в ЛМС НИУ ВШЭ} & \makecell[lt]{15.04.2025- \\20.04.2025} \\
                \cline{3-4}
                & & \makecell[lt]{Защита программного продукта \\комиссии} & \makecell[lt]{20.04.2025- \\25.04.2025} \\
                \hline
            \end{tabular}
        }
    \end{table}


    \section{Порядок контроля и приемки}

    \subsection{Виды испытаний}
    Проверка программного продукта, в том числе и на соответствие техническому заданию, осуществляется исполнителем вместе с заказчиком согласно «Программе и методике испытаний», а также пункту 5.2.

    \subsection{Общие требования к приемке работы}
    Защита выполненного проекта осуществляется комиссии, состоящей из преподавателей департамента программной инженерии, в утверждённые приказом декана ФКН сроки.

    Контроль и приемка разработки осуществляются в соответствии с документом <<Программа и методика испытаний <<Веб-приложение для ведения лабораторных и вычислительных экспериментов с поддержкой онтологий\unskip>>.
    \newpage

    \printbibliography[title=Список источников, heading=bibintoc]
    \newpage

    \addition{Используемые понятия и определения}{terms}
    \phantomsection
\addcontentsline{toc}{section}{Основные определения, термины и сокращения}

\begin{description}[style=unboxed,leftmargin=0cm]

    \item[\textbf{Валидность}] -- степень соответствия полученных данных реальности.

    \item[\textbf{Вычислительный эксперимент}] -- метод исследования, при котором математические модели и компьютерные расчёты используются для анализа сложных систем и прогнозирования их поведения.

    \item[\textbf{Домен (предметная область)}] -- область знаний, к которой относится конкретное программное обеспечение.

    \item[\textbf{Интероперабельность}] -- возможность обмена и понимания данных между разными системами на основе их смыслового значения.

    \item[\textbf{Клиент-серверная архитектура}] -- принцип проектирования ПО, согласно которому клиент запрашивает данные у сервера, который их обрабатывает и отправляет обратно.

    \item[\textbf{Лабораторный эксперимент}] -- метод научного исследования, проводимый в контролируемых условиях лаборатории для проверки гипотез, изучения свойств объектов или выявления закономерностей.

    \item[\textbf{Онтология}] -- формализованное представление знаний в определённой предметной области, включающее определения понятий, их свойства и взаимосвязи между ними. Используется в искусственном интеллекте, базах знаний и прочих областях.

    \item[\textbf{Слоистая архитектура}] -- принцип проектирования ПО, согласно которому система разделена на слои, каждый из которых выполняет свою роль.

    \item[\textbf{СУБД (Система управления базами данных)}] -- программное обеспечение для создания, управления и работы с базами данных.

    \item[\textbf{API (Application Programming Interface)}] -- интерфейс программирования приложений, набор правил и инструментов для взаимодействия между программными компонентами.

    \item[\textbf{ELN (Electronic Laboratory Notebook)}] -- электронный лабораторный журнал, который используется для ведения, хранения и управления научными записями, заменяя традиционные бумажные записи.

    \item[\textbf{JWT (JSON Web Token)}] -- компактный и безопасный способ передачи информации между сторонами в виде JSON-объекта, часто используется для аутентификации.

    \item[\textbf{ORM (Object-Relational Mapping)}] -- технология, позволяющая работать с базами данных с использованием объектно-ориентированных подходов вместо SQL-запросов.

    \item[\textbf{SPA (Single Page Application)}] -- веб-приложение, которое загружается один раз и обновляет контент без полной перезагрузки страницы, улучшая пользовательский опыт.

\end{description}
    \newpage

    \listRegistration

\end{document} % конец документа