\setcounter{section}{1}

\sectionVKR{Предметная область и существующие решения}
В данной главе рассматриваются современные подходы к разработке электронных лабораторных журналов (ELN) и их использование в научных исследованиях.
Особое внимание уделяется проблеме стандартизации и интероперабельности данных, а также применению онтологий для решения этой задачи.
Приводится обзор существующих решений, анализируются их преимущества и недостатки, а также формулируется общая проблема, которую призвана решить данная работа.

\subsection{Описание предметной области}

Электронные лабораторные журналы (ELN) представляют собой цифровые системы, предназначенные для ведения научных записей, хранения экспериментальных данных и их последующего анализа.
В отличие от традиционных бумажных лабораторных журналов, ELN обеспечивают удобный поиск информации, совместную работу исследователей и интеграцию с различными внешними источниками данных, такими как базы научных публикаций, лабораторное оборудование и онтологические системы знаний~\cite{ontology:base2}.

Одним из ключевых аспектов ELN является работа с измерительными данными, которые представляют собой числовые значения, полученные в ходе экспериментов.
Эти данные могут быть выражены в различных единицах измерения, а также сопровождаться метаинформацией о методе их получения и условиях эксперимента.
Для корректного хранения и интерпретации таких данных важно использовать стандартизированные подходы, обеспечивающие их точность и воспроизводимость.

В этом контексте онтологии играют важную роль в обеспечении стандартизации научных данных.
Онтологии позволяют унифицировать представление информации, связывая термины и концепции с четко определенными значениями и отношениями.
Например, онтология OM2 предоставляет формализованные определения единиц измерения и их взаимосвязей, а ChEBI описывает химические соединения и их свойства.
Интеграция таких онтологий в ELN позволяет не только структурировать данные, но и обеспечивать их интероперабельность между различными системами.

\subsection{Описание проблем и особенностей}

Несмотря на очевидные преимущества ELN, существует ряд проблем, которые затрудняют их эффективное использование в научных исследованиях.
Одной из ключевых проблем является отсутствие единообразного подхода к структурированию экспериментальных данных.
В разных лабораториях и научных группах используются различные форматы записи данных, что затрудняет их совместное использование и автоматизированную обработку.

Еще одной важной проблемой является отсутствие поддержки онтологий в большинстве существующих ELN. Хотя некоторые решения позволяют добавлять аннотации к данным, они не заставляют делать это обязательно и, тем более, использовать элементы онтологий.
Например, при вводе данных о размере объекта исследователь может записать его в сантиметрах, в то время как другая система ожидает значение в метрах.
Без привязки к онтологии, такой как OM2, невозможно автоматически определить, что значения эквивалентны и не требуют приведения к единому формату.

Также существует проблема удобства работы с ELN. Многие системы перегружены функционалом, сложны в освоении и не обладают интуитивно понятным интерфейсом.
В результате исследователи предпочитают использовать традиционные электронные таблицы (например, Excel), несмотря на их ограниченные возможности по структурированному хранению данных и обеспечению их целостности.

\subsection{Описание существующих решений}

На сегодняшний день существует ряд электронных лабораторных тетрадей (ELN), предназначенных для ведения научных записей, управления экспериментальными данными и их последующего анализа.
Однако ни одно из существующих решений не реализует полную поддержку онтологий, необходимую для стандартизации и автоматизированной интерпретации данных.
Ниже рассмотрены наиболее распространенные ELN с выделением их ключевых характеристик.

\subsubsection{Benchling}
\textit{Benchling}~\cite{ELN:Benchling} — облачное ELN-решение, ориентированное на работу с биологическими и химическими данными.
Оно широко используется в биотехнологических компаниях, академических лабораториях и фармацевтической промышленности.

\begin{enumerate}
    \item Стоимость использования: бесплатная версия с ограниченным функционалом, коммерческая лицензия — от нескольких тысяч долларов в год.
    \item Функциональность: ведение лабораторных записей, управление молекулярными структурами, совместная работа, интеграция с лабораторным оборудованием.
    \item Поддержка онтологий: отсутствует, данные вводятся в произвольном формате без формальной привязки к единицам измерения или терминам.
    \item Автоматическая обработка данных: отсутствует, анализ данных требует ручной проверки и интерпретации.
    \item Преимущества: облачная архитектура, интеграция с биоинформатическими инструментами, поддержка командной работы.
    \item Недостатки: отсутствие онтологической поддержки, невозможность автоматизированной интерпретации данных, ориентированность в первую очередь на биологические исследования.
\end{enumerate}

\subsubsection{LabArchives}

\textit{LabArchives}~\cite{ELN:LabArchives} — облачная электронная лабораторная тетрадь, используемая в академической среде и промышленности.
Она поддерживает ведение записей, управление файлами и контроль версий, что делает ее удобным инструментом для научных коллективов.

\begin{enumerate}
    \item Стоимость использования: подписка от \$15 в месяц на пользователя, есть академические скидки, но возможны ограничения при оплате подписки.
    \item Функциональность: ведение записей об экспериментах, прикрепление файлов, контроль версий, совместная работа.
    \item Поддержка онтологий: отсутствует, данные хранятся в виде текстовых и табличных записей без привязки к семантическим структурам.
    \item Автоматическая обработка данных: отсутствует, анализ данных требует ручной проверки и интерпретации.
    \item Преимущества: облачная архитектура, поддержка коллективной работы, удобный поиск по записям.
    \item Недостатки: отсутствие онтологической поддержки, невозможность автоматизированной интерпретации данных.
\end{enumerate}

\subsubsection{OpenBIS}

\textit{OpenBIS}~\cite{ELN:OpenBIS} — мощная платформа для управления научными данными, разработанная в Швейцарском федеральном технологическом институте (ETH Zurich).
Она предназначена для хранения, обработки и анализа больших объемов экспериментальных данных.

\begin{enumerate}
    \item Стоимость использования: бесплатно (open-source), но требует развертывания и настройки.
    \item Функциональность: хранение и структурирование данных, интеграция с лабораторным оборудованием, автоматизация процессов обработки.
    \item Поддержка онтологий: отсутствует, данные организуются в соответствии с пользовательскими схемами, но не имеют строгой семантической интерпретации.
    \item Автоматическая обработка данных: частично поддерживается, но отсутствует механизм онтологической проверки корректности единиц измерения.
    \item Преимущества: высокая масштабируемость, поддержка сложных моделей данных, интеграция с лабораторными приборами.
    \item Недостатки: сложная настройка, высокая стоимость развертывания (инфраструктура и специалисты), отсутствие полноценной онтологической поддержки, невозможность автоматизированной интерпретации данных.
\end{enumerate}

\subsubsection{Chemotion ELN}

\textit{Chemotion ELN}~\cite{ELN:Chemotion} — специализированная электронная лабораторная тетрадь для химиков, разработанная Институтом технологий Карлсруэ (KIT).
Она позволяет работать с химическими соединениями, реакциями и аналитическими данными.

\begin{enumerate}
    \item Стоимость использования: бесплатно, но требует установки и настройки на собственном сервере.
    \item Функциональность: ведение лабораторных записей, работа с молекулярными структурами, управление химическими соединениями.
    \item Поддержка онтологий: отсутствует, данные представлены в виде таблиц и химических структур без привязки к внешним онтологиям.
    \item Автоматическая обработка данных: частично поддерживается, но отсутствует механизм онтологической проверки корректности единиц измерения.
    \item Преимущества: специализированные инструменты для работы с химическими соединениями, интеграция с внешними химическими базами данных.
    \item Недостатки: ограниченный функционал за пределами химических исследований, отсутствие поддержки онтологий для стандартизации данных.
\end{enumerate}

\subsection{Описание разрабатываемого решения}

В этом разделе представлены функциональные и нефунциональные требования к разрабатываемому сервису.

Разрабатываемое веб-приложение должно обеспечивать удобное управление и работу с экспериментами, поддержку работы с онтологиями, а также интеграцию с внешними системами.

Приложение должно позволять пользователям создавать, редактировать и удалять эксперименты, добавлять к ним описание, указывать данные и метаданные (например, дату проведения, автора, связанные проекты или эксперименты) для вычислительного эксперимента и управлять таблицей измерений для лабораторного.
Эксперименты должны группироваться по папкам, обеспечивая удобную навигацию.

Таблицы измерений являются ключевым элементом системы, поэтому пользователи должны иметь возможность добавлять, удалять и редактировать строки и столбцы, выбирать тип данных (числовые значения, текст, даты) и загружать данные из внешних источников в формате JSON~\cite{Format:JSON}.
Также необходимо предусмотреть экспорт данных в форматы JSON и XML~\cite{Format:XML}.

Ключевой функцией системы является интеграция с онтологиями.
Пользователи должны иметь возможность привязывать столбцы таблицы к онтологиям, что обеспечит строгую стандартизацию данных.
Информация об онтологиях должна загружаться из базы данных Neo4j~\cite{DB:Neo4j}, поддерживая возможность обновления и расширения перечня онтологических знаний.

Для обеспечения совместной работы пользователей необходимо реализовать систему аутентификации и авторизации с разграничением ролей.
Должен быть заложен механизм различных уровней доступа, включая администратора и исследователя.

Для интеграции с внешними системами, расширения возможностей приложения или ручного анализа данных необходимо реализовать экспорт данных в удобочитаемом (JSON) и стандартизованном (XML) форматах, а также обеспечить поддержку REST API~\cite{arch:REST}.

Приложение должно быть построено по клиент-серверной архитектуре, где фронтенд разрабатывается на Vue.js~\cite{Framework:VueJS}, а серверная часть — на FastAPI~\cite{Framework:FastAPI}.
В качестве хранилища данных используется PostgreSQL~\cite{DB:PostgreSQL} для хранения информации о пользователях, экспериментах и ролях, а Neo4j применяется для работы с онтологиями.
Важно обеспечить высокую производительность системы, чтобы взаимодействие с таблицами и онтологиями происходило быстро, а также реализовать меры безопасности, включая аутентификацию с использованием JWT~\cite{Security:JWT} и защиту персональных данных пользователей.

Такое приложение позволит автоматизировать работу с экспериментальными данными, минимизировать ошибки, связанные с некорректными единицами измерения, и упростить интеграцию с другими системами, делая работу исследователей более эффективной.

\subsection{Сравнительный анализ}

В таблице~\ref{tab:comparison} представлено сравнение ключевых характеристик существующих решений для электронных лабораторных тетрадей (ELN) и разрабатываемого сервиса. Сравниваются такие параметры, как стоимость использования, поддержка онтологий, возможность автоматической проверки единиц измерения, экспорт и импорт данных, поддержка REST API, разграничение ролей и используемые базы данных. Из таблицы видно, что разрабатываемый сервис превосходит большинство аналогов по ряду ключевых функций, таких как поддержка онтологий и автоматическая проверка единиц измерения, а также предлагает свободный доступ, что делает его более привлекательным для пользователей.

\begin{table}[h]
    \centering
    \resizebox{\textwidth}{!}{
        \begin{tabular}{|l|c|c|c|c|c|}
            \hline
            \textbf{Характеристика}  & \textbf{OpenBIS}               & \textbf{LabArchives}          & \textbf{Benchling}                   & \textbf{Chemotion ELN}               & \textbf{Разрабатываемый сервис}             \\
            \hline
            Стоимость использования  & \cellcolor{green!20}Бесплатно  & \cellcolor{red!20}Подписка    & \cellcolor{red!20}Подписка           & \cellcolor{green!20}Бесплатно        & \cellcolor{green!20}Бесплатно               \\
            \hline
            Поддержка онтологий      & \cellcolor{red!20}--           & \cellcolor{red!20}--          & \cellcolor{red!20}--                & \cellcolor{red!20}--                & \cellcolor{green!20}+                      \\
            \hline
            \makecell[lt]{Автоматическая проверка единиц измерения \\при сравнении экспериментов}    & \cellcolor{red!20}-- & \cellcolor{red!20}--         & \cellcolor{red!20}--                & \cellcolor{red!20}--                & \cellcolor{green!20}+                      \\
            \hline
            Экспорт/импорт данных       & \cellcolor{yellow!20}CSV, JSON          & \cellcolor{yellow!20}CSV, PDF         & \cellcolor{green!20}CSV, JSON, PDF   & \cellcolor{yellow!20}CSV, JSON       & \cellcolor{yellow!20}JSON, XML          \\
            \hline
            Поддержка REST API      & \cellcolor{green!20}+          & \cellcolor{green!20}+         & \cellcolor{green!20}+               & \cellcolor{green!20}+               & \cellcolor{green!20}+                      \\
            \hline
            Разграничение ролей       & \cellcolor{green!20}+           & \cellcolor{green!20}+          & \cellcolor{green!20}+               & \cellcolor{green!20}+               & \cellcolor{yellow!20}+-                      \\
            \hline
            Хранение онтологий              & \cellcolor{red!20}-- & \cellcolor{red!20}--          & \cellcolor{red!20}--                & \cellcolor{red!20}--                & \cellcolor{green!20}+              \\
            \hline
            База данных & \cellcolor{green!20}PostgreSQL       & \cellcolor{red!20}--          & \cellcolor{green!20}PostgreSQL       & \cellcolor{red!20}--                 & \cellcolor{green!20}PostgreSQL + Neo4j      \\
            \hline
            Основной стек технологий & \cellcolor{green!20}Java       & \cellcolor{red!20}--          & \cellcolor{red!20}--               & \cellcolor{green!20}JavaScript, Ruby & \cellcolor{green!20}FastAPI, Vue.js    \\
            \hline
        \end{tabular}
    }
    \caption{Сравнение существующих ELN-решений и разрабатываемого сервиса}
    \label{tab:comparison}
\end{table}

\anonsubsection{Выводы по главе}

В ходе анализа существующих решений для электронных лабораторных тетрадей (ELN) и разрабатываемого сервиса можно выделить несколько ключевых преимуществ.
В отличие от LabArchives и Benchling, которые требуют подписки, сервис будет бесплатным, что делает его доступным для широкого круга пользователей.
Кроме того, разрабатываемое приложение поддерживает онтологии, полной поддержки и обязательного использования которых нет в рассматриваемых решениях.
Это позволяет обеспечивать строгую стандартизацию данных, тем самым предотвращая ошибки и повышая точность данных.
Экспорт и импорт данных поддерживаются в популярных форматах, таких как JSON и XML, а также, что позволяет легко обмениваться данными с другими системами.

Все рассмотренные аналоги поддерживают базовую функциональность REST API и разграничение ролей.
Разрабатываемое приложение, в свою очередь, включает хранение онтологий в графовой СУБД Neo4j, что повышает гибкость и эффективность работы с онтологиями.
Технологический стек сервиса, включающий FastAPI для серверной части и Vue.js для клиентской, обеспечивает высокую производительность и удобство в использовании.

Таким образом, разрабатываемое приложение обладает рядом преимуществ, включая свободность пользования, обязательность использования онтологий, а также современный технологический стек, что делает его более эффективным и доступным инструментом для научных исследований.