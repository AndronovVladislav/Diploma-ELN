\anonsection{\centering Введение}

Современные научные исследования требуют систематического подхода к хранению, обработке и анализу экспериментальных данных. Традиционные методы ведения лабораторных записей, основанные на бумажных журналах или неструктурированных электронных документах, затрудняют поиск информации, совместное использование данных и уменьшают их валидность. В связи с этим все большую популярность приобретают электронные лабораторные журналы, которые позволяют организованно фиксировать результаты экспериментов и облегчать их последующую обработку.

Одной из ключевых проблем, возникающих при работе с ELN, является стандартизация данных. Исследователи могут использовать разные системы единиц измерения, а также различные обозначения одних и тех же сущностей, что затрудняет интерпретацию и сравнение данных. Решением этой проблемы является использование онтологий — формализованных моделей знаний, описывающих объекты и их взаимосвязи в определенной предметной области.

В данной работе рассматривается разработка веб-приложения, реализующего концепцию ELN с возможностью привязки экспериментальных данных к онтологиям. Это позволит не только систематизировать данные, но и повысить их интероперабельность, а также упростить интеграцию с другими научными ресурсами. В качестве онтологических источников предлагается использование стандартных онтологий, таких как OM2 (Ontology of Units of Measure) и ChEBI (Chemical Entities of Biological Interest), которые обеспечивают унификацию единиц измерения и химических сущностей.

\paragraph*{Цель:} разработка веб-приложения для ведения лабораторных записей с поддержкой онтологий, что обеспечит стандартизацию данных и упростит их обработку.

\paragraph*{Задачи:}
\begin{enumerate}
    \item Анализ предметной области и существующих решений
    \item Проектирование архитектуры системы
    \begin{enumerate}[label=\arabic{enumi}.\arabic*.]
        \item Разработка общей архитектуры веб-приложения с учетом клиент-серверного взаимодействия
        \item Определение структуры хранения данных
        \item Проектирование REST API для работы с онтологиями и управления экспериментами
    \end{enumerate}
    \item Реализация серверной части приложения
    \item Реализация клиентской части приложения
    \item Интеграция с онтологиями
    \item Тестирование и валидация работы системы
    \item Документирование и развертывание системы
\end{enumerate}