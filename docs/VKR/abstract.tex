\section*{\centering Реферат}
В современном мире научные исследования становятся всё более сложными и требуют эффективных инструментов для документирования, анализа и обмена данными.
Одним из таких инструментов является электронный лабораторный журнал (сокращённо ELN), который позволяет учёным структурировать данные экспериментов, обеспечивать их воспроизводимость и стандартизацию.
В данной работе предлагается разработка веб-приложения, реализующего функционал ELN, основное внимание уделяется возможности привязки данных эксперимента к онтологиям (например, OM2~\cite{ontology:OM2} или ChEBI~\cite{ontology:СhEBI}), что позволяет унифицировать единицы измерения и улучшить интероперабельность данных, а также созданию легковесного интерфейса без лишних функций, что должно облегчить и ускорить работу.
Разработанное решение может быть использовано в научных лабораториях, исследовательских центрах и образовательных учреждениях для повышения качества документирования экспериментальных данных, но в первую очередь предназначено для РТУ МИРЭА.

Использование онтологий в научных исследованиях обосновано их способностью обеспечивать семантическую совместимость данных, что особенно важно при анализе и обмене результатами между различными лабораториями и информационными системами.
Согласно работе \textit{Relations in biomedical ontologies}~\cite{ontology:base1}, онтологии позволяют стандартизировать представление данных, что делает их пригодными для автоматизированной обработки.
В то же время, как отмечается в работе \textit{Knowledge Graphs}~\cite{ontology:base2}, применение графовых баз данных значительно упрощает управление онтологиями и их интеграцию в прикладные системы, что повышает эффективность работы с научными данными.

Работа содержит 56 страницу, 10 рисунков, 1 таблицу, 41 источник, 3 листинга.

\textit{\textbf{Ключевые слова -- онтологически-контролируемое управление данными; электронный лабораторный журнал; управление экспериментами; онтология предметной области}}

\newpage

\section*{\centering Abstract}

In the modern world, scientific research is becoming increasingly complex and demands effective tools for documentation, analysis, and data sharing.
One such tool is the electronic laboratory notebook (ELN), which enables researchers to structure experimental data, ensure its reproducibility, and enforce standardization.
In this work, I propose the development of a web application implementing ELN functionality, with a particular focus on the ability to link experimental data to ontologies (for example, OM2~\cite{ontology:OM2} or ChEBI~\cite{ontology:СhEBI}), thereby unifying measurement units and improving data interoperability, as well as creating a lightweight user interface without excess functions, which should make it easier and faster to work.
The developed solution can be used in scientific laboratories, research centers, and educational institutions to enhance the quality of experimental data documentation, but is primarily intended for RTU MIREA.

The use of ontologies in scientific research is justified by their ability to provide semantic compatibility of data, which is especially important when analyzing and sharing results between different laboratories and information systems.
According to the study \textit{Relations in biomedical ontologies}\cite{ontology:base1}, ontologies allow for standardized data representation, making them suitable for automated processing.
At the same time, as noted in \textit{Knowledge Graphs}\cite{ontology:base2}, the application of graph databases significantly simplifies the management of ontologies and their integration into applied systems, thereby increasing the efficiency of working with scientific data.

The work contains 56 pages, 10 diagrams, 1 tables, 41 figures, 3 listings.

\textit{\textbf{Keywords -- ontologically-controlled data management; electronic laboratory notebook; experiment management; domain ontology}}
