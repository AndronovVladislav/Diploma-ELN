\section*{\centering Реферат}
В современном мире научные исследования становятся всё более сложными и требуют эффективных инструментов для документирования, анализа и обмена данными. Одним из таких инструментов является электронный лабораторный журнал (сокращённо ELN), который позволяет учёным структурировать данные экспериментов, обеспечивать их воспроизводимость и стандартизацию. В данной работе предлагается разработка веб-приложения, реализующего функционал ELN, основное внимание уделяется возможности привязки данных эксперимента к онтологиям (например, OM2 \cite{ontology:OM2} или ChEBI \cite{ontology:СhEBI}), что позволяет унифицировать единицы измерения и улучшить интероперабельность данных. Разработанное решение может быть использовано в научных лабораториях, исследовательских центрах и образовательных учреждениях для повышения качества документирования экспериментальных данных, но в первую очередь предназначено для РТУ МИРЭА.

Использование онтологий в научных исследованиях обосновано их способностью обеспечивать семантическую совместимость данных, что особенно важно при анализе и обмене результатами между различными лабораториями и информационными системами. Согласно работе \textit{Relations in biomedical ontologies} \cite{ontology:base1}, онтологии позволяют стандартизировать представление данных, что делает их пригодными для автоматизированной обработки. В то же время, как отмечается в работе \textit{Knowledge Graphs} \cite{ontology:base2}, применение графовых баз данных значительно упрощает управление онтологиями и их интеграцию в прикладные системы, что повышает эффективность работы с научными данными.

Работа содержит 48 страниц, 10 рисунков, 1 таблицу, 40 источников.

\textit{\textbf{Ключевые слова -- онтологически-контролируемое управление данными; электронный лабораторный журнал; управление экспериментами; онтология предметной области}}

\newpage

\section*{\centering Abstract}
In today's world, scientific research is becoming increasingly complex and requires effective tools for documenting, analyzing, and sharing data. One of these tools is the electronic laboratory journal (abbreviated as ELN), which allows scientists to structure experimental data, ensure their reproducibility and standardization. This paper proposes the development of a web application that implements the ELN functionality, focusing on the possibility of linking experimental data to ontologies (for example, OM2\cite{ontology:OM2} or ChEBI \cite{ontology:СhEBI}), which makes it possible to unify units of measurement and improve data interoperability. The developed solution can be used in scientific laboratories, research centers and educational institutions to improve the quality of documentation of experimental data, but is primarily intended for RTU MIREA.

The use of ontologies in scientific research is justified by their ability to ensure semantic compatibility of data, which is especially important when analyzing and exchanging results between different laboratories and information systems. According to the work \textit{Relations in biomedical ontologies} \cite{ontology:base1}, ontologies allow you to standardize the representation of data, which makes them suitable for automated processing. At the same time, as noted in \textit{Knowledge Graphs} \cite{ontology:base2}, the use of graph databases greatly simplifies the management of ontologies and their integration into application systems, which increases the efficiency of working with scientific data.

The work contains 48 pages, 10 diagrams, 1 tables, 40 sources.

\textit{\textbf{Keywords -- ontologically-controlled data management; electronic laboratory notebook; experiment management; domain ontology}}
