\setcounter{section}{3}
\setcounter{subsection}{0}
\sectionVKR{Программная реализация} \label{sec: 3}

\subsection{Общая архитектура клиент-серверного приложения}

Приложение построено по клиент-серверной архитектуре, где фронтенд и бэкенд взаимодействуют через REST API. Данное решение позволяет обеспечивать масштабируемость, гибкость разработки и упрощённую интеграцию с различными внешними сервисами, включая онтологии OM2 и СhEBI.

На верхнем уровне приложение разделяется на три основных слоя:
\begin{enumerate}
    \item Клиентская часть — реализована на Vue.js с использованием PrimeVue, Pinia для управления состоянием и Vue Router для маршрутизации.
    \item Серверная часть — написана на FastAPI, использует SQLAlchemy~\cite{Library:SQLAlchemy} для работы с базой данных, Alembic~\cite{Library:Alembic} для управления миграциями и JWT для аутентификации.
    \item Базы данных — в качестве реляционной СУБД используется PostgreSQL в которой хранятся все данные пользователей, экспериментов и привязок к онтологиям, а в качестве графовой – Neo4j для хранения полной информации о доступных для использованию онтологиях.
\end{enumerate}

% Общая схема архитектуры представлена на рисунке~\ref{fig:architecture}.

% \begin{figure}[H]
%     \centering
%     \includegraphics[width=0.6\linewidth]{img/architecture-diagram.png}
%     \caption{Общая архитектура приложения}
%     \label{fig:architecture}
% \end{figure}

\subsection{Архитектурные паттерны и подходы}

Для структурирования кода и повышения удобочитаемости используются следующие архитектурные паттерны:

\begin{enumerate}
    \item MVVM (Model-View-ViewModel) — используется во Vue.js для упрощения двустороннего связывания данных между представлением и логикой.
    \item ORM (Object-Relational Mapping) — SQLAlchemy используется как ORM для работы с реляционной базой данных PostgreSQL.
    \item RESTful API — бэкенд реализует REST API, обеспечивающий взаимодействие с клиентской частью и внешними сервисами.
\end{enumerate}

\subsection{Выбор технологий}

Разработка велась с использованием следующих технологий:
\begin{enumerate}
    \item Backend: FastAPI, SQLAlchemy, Alembic, Pydantic, JWT.
    \item Frontend: Vue.js, PrimeVue, Pinia, Vue Router, Axios~\cite{Library:Axios}.
    \item Database: PostgreSQL.
    \item DevOps: Docker~\cite{Tool:Docker}, Nginx~\cite{Tool:Nginx}, GitHub Actions~\cite{Tool:GitHubActions} для тестов.
\end{enumerate}

\subsection{Разработка серверной части}

\subsubsection{Структура серверного приложения}

Кодовая база серверной части организована следующим образом:

\begin{enumerate}
    \item \texttt{alembic} — пакет с настройками инструмента Alembic и миграциями базы данных.
    \item \texttt{models} — пакет с ORM-моделями базы данных, описанные с использованием SQLAlchemy, а также кодом для работы с онтологиями.
    \item \texttt{routes} — пакет с обработчиками HTTP-запросов, организованными в соответствии с REST API.
    \item \texttt{services} — пакет с бизнес-логикой приложения, отделённой от маршрутов.
    \item \texttt{schemas} — пакет Pydantic-схемы для валидации входных и выходных данных API.
    \item \texttt{test} – пакет с тестами.
    \item \texttt{config.py} — файл конфигурации приложения, загружающий переменные окружения и производящий прочие настройки.
\end{enumerate}

% \todo[inline]{Сделать диаграмму пакетов}

% Общая структура серверного приложения представлена на рисунке~\ref{fig:backend-structure}.

% \begin{figure}[H]
%     \centering
%     \includegraphics[width=0.6\linewidth]{backend-structure.png}
%     \caption{Структура серверной части}
%     \label{fig:backend-structure}
% \end{figure}

\subsubsection{Модели базы данных}\label{ORM models}

Данные в приложении хранятся в PostgreSQL. Для работы с базой данных используется ORM SQLAlchemy. Основные сущности базы данных:

\begin{enumerate}
    \item \texttt{User} — хранит данные о пользователях, включая имя пользователя, хешированный пароль и роль (администратор или исследователь).
    \item \texttt{Profile} — расширяет информацию о пользователе, включая имя, фамилию и дату регистрации.
    \item \texttt{Experiment} — базовая таблица для всех экспериментов, содержащая описание, путь к файлу и временные метки.
    \item \texttt{LaboratoryExperiment} — конкретизация \texttt{Experiment}, привязанная к пользователю и содержащая измерения и описание столбцов.
    \item \texttt{ComputationalExperiment} — вычислительный эксперимент, связанный с шаблоном и входными/выходными данными.
    \item \texttt{Measurement} — хранит данные измерений, относящиеся к лабораторному эксперименту.
    \item \texttt{Ontology} — содержит данные о доступных онтологиях, их названия и метки.
    \item \texttt{ColumnDescription} — описание столбцов эксперимента с привязкой к онтологиям.
    \item \texttt{Schema} — хранит JSON-данные входов, выходов, параметров и контекста эксперимента.
    \item \texttt{SchemaLinkedTable} — абстрактная таблица для связи схем с другими таблицами.
    \item \texttt{ComputationalExperimentTemplate} — шаблон вычислительного эксперимента, содержащий связанные эксперименты.
    \item \texttt{ComputationalExperimentData} — данные вычислительного эксперимента, связанные с конкретным экспериментом.
\end{enumerate}

\subsubsection{Реализация API для работы с экспериментами}

API реализовано в соответствии с RESTful-архитектурой. Основные эндпоинты:

\paragraph{Эндпоинты аутентификации}

\begin{enumerate}
    \item \texttt{POST /auth/signup} — регистрация нового пользователя.
    \begin{enumerate}[label=\arabic{enumi}.\arabic*.]
        \item Входные данные: \texttt{UserSignup} (имя пользователя, пароль).
        \item Ответ: сообщение о регистрации.
    \end{enumerate}

    \item \texttt{POST /auth/login} — вход в систему.
    \begin{enumerate}[label=\arabic{enumi}.\arabic*.]
        \item Входные данные: \texttt{UserLogin} (имя пользователя, пароль).
        \item Ответ: объект \texttt{UserResponse}, содержащий \texttt{access\_token} и \texttt{refresh\_token}.
    \end{enumerate}

    \item \texttt{POST /auth/refresh} — обновление access-токена по refresh-токену.
    \begin{enumerate}[label=\arabic{enumi}.\arabic*.]
        \item Входные данные: refresh-токен, переданный в теле запроса.
        \item Ответ: новый \texttt{access\_token} и \texttt{refresh\_token}.
    \end{enumerate}
\end{enumerate}

\paragraph{Эндпоинты управления экспериментами}

\begin{enumerate}
    \item \texttt{GET /experiment/} — получение списка экспериментов пользователя.
    \begin{enumerate}[label=\arabic{enumi}.\arabic*.]
        \item Входные параметры: \texttt{username} (имя пользователя), \texttt{desired\_keys} (список ключей).
        \item Ответ: список словарей с данными экспериментов.
    \end{enumerate}

    \item \texttt{GET /experiment/\{experiment\_id\}} — получение данных конкретного эксперимента.
    \begin{enumerate}[label=\arabic{enumi}.\arabic*.]
        \item Входные параметры: \texttt{experiment\_id} (идентификатор эксперимента), \texttt{kind} (тип эксперимента: лабораторный или вычислительный).
        \item Ответ: данные лабораторного или вычислительного эксперимента.
    \end{enumerate}

    \item \texttt{POST /experiment/import} — импорт эксперимента.
    \begin{enumerate}[label=\arabic{enumi}.\arabic*.]
        \item Входные данные: объект \texttt{ExperimentDescription}.
        \item Ответ: словарь, содержащий описание столбцов эксперимента.
    \end{enumerate}

    \item \texttt{POST /experiment/} — создание нового эксперимента.
    \begin{enumerate}[label=\arabic{enumi}.\arabic*.]
        \item Входные данные: объект \texttt{ExperimentCreate} (название, описание, пользователь).
        \item Ответ: объект \texttt{ExperimentRead} с ID и атрибутами созданного эксперимента.
    \end{enumerate}

    \item \texttt{PUT /experiment/\{experiment\_id\}} — обновление данных эксперимента.
    \begin{enumerate}[label=\arabic{enumi}.\arabic*.]
        \item Входные параметры: \texttt{experiment\_id} (идентификатор эксперимента).
        \item Входные данные: объект \texttt{ExperimentUpdate} (новое описание, новые параметры).
        \item Ответ: обновлённый объект \texttt{ExperimentRead}.
    \end{enumerate}

    \item \texttt{DELETE /experiment/\{experiment\_id\}} — удаление эксперимента.
    \begin{enumerate}[label=\arabic{enumi}.\arabic*.]
        \item Входные параметры: \texttt{experiment\_id} (идентификатор эксперимента).
        \item Ответ: сообщение об успешном удалении.
    \end{enumerate}
\end{enumerate}

\paragraph{Эндпоинты управления пользователями}

\begin{enumerate}
    \item \texttt{GET /users/} — получение списка пользователей.
    \begin{enumerate}[label=\arabic{enumi}.\arabic*.]
        \item Ответ: список пользователей с базовой информацией.
    \end{enumerate}

    \item \texttt{GET /users/\{user\_id\}} — получение информации о конкретном пользователе.
    \begin{enumerate}[label=\arabic{enumi}.\arabic*.]
        \item Входные параметры: \texttt{user\_id} (идентификатор пользователя).
        \item Ответ: объект \texttt{UserResponse}, содержащий имя, роль и мета-данные.
    \end{enumerate}

    \item \texttt{PUT /users/\{user\_id\}} — обновление данных пользователя.
    \begin{enumerate}[label=\arabic{enumi}.\arabic*.]
        \item Входные параметры: \texttt{user\_id} (идентификатор пользователя).
        \item Входные данные: объект \texttt{UserUpdate} (новые параметры профиля).
        \item Ответ: обновлённый объект \texttt{UserResponse}.
    \end{enumerate}

    \item \texttt{DELETE /users/\{user\_id\}} — удаление пользователя.
    \begin{enumerate}[label=\arabic{enumi}.\arabic*.]
        \item Входные параметры: \texttt{user\_id} (идентификатор пользователя).
        \item Ответ: сообщение об успешном удалении.
    \end{enumerate}
\end{enumerate}

\paragraph{Эндпоинты работы с онтологиями}

\begin{enumerate}
    \item \texttt{GET /ontology/} — получение списка доступных онтологий.
    \begin{enumerate}[label=\arabic{enumi}.\arabic*.]
        \item Ответ: список объектов \texttt{Ontology}, содержащих названия и идентификаторы онтологий.
    \end{enumerate}

    \item \texttt{POST /ontology/} — добавление новой онтологии.
    \begin{enumerate}[label=\arabic{enumi}.\arabic*.]
        \item Входные данные: объект \texttt{OntologyCreate} (название, описание, формат).
        \item Ответ: объект \texttt{OntologyRead} с ID новой онтологии.
    \end{enumerate}

    \item \texttt{GET /ontology/\{ontology\_id\}} — получение информации о конкретной онтологии.
    \begin{enumerate}[label=\arabic{enumi}.\arabic*.]
        \item Входные параметры: \texttt{ontology\_id} (идентификатор онтологии).
        \item Ответ: объект \texttt{OntologyRead}, содержащий детали онтологии.
    \end{enumerate}

    \item \texttt{DELETE /ontology/\{ontology\_id\}} — удаление онтологии.
    \begin{enumerate}[label=\arabic{enumi}.\arabic*.]
        \item Входные параметры: \texttt{ontology\_id} (идентификатор онтологии).
        \item Ответ: сообщение об успешном удалении.
    \end{enumerate}
\end{enumerate}

\subsubsection{Аутентификация и авторизация пользователей}

Для управления доступом используется аутентификация на основе JWT. Пользователь после успешного входа получает короткоживущий access-токен, позволяющий получить доступ к ресурсам системы, и refresh-токен, в течение жизни которого можно перевыпустить access-токен. Если срок жизни refresh-токена истекает, требуется повторная авторизация с указанием имени пользователя и пароля.

При каждом запросе к защищённым маршрутам проверяется валидность токена.

\subsubsection{Работа с онтологиями}

Для работы с онтологиями используется интеграция с OM2 и СhEBI. Приложение позволяет привязывать столбцы таблиц экспериментов к онтологиям, что упрощает интерпретацию данных.

\begin{enumerate}
    \item Поддерживается хранение привязок к онтологиям через таблицу \texttt{ColumnDescription}.
    \item Используется Neo4j для представления связей между сущностями онтологий.
    \item Запросы к онтологиям выполняются с использованием языка Cypher (пример в листинге \ref{lst:Cypher0}).
\end{enumerate}

\begin{lstlisting}[frame=single, basicstyle=\footnotesize\ttfamily, label={lst:Cypher0}, caption={Пример запроса к Neo4j для поиска онтологических связей},captionpos=b, breaklines=true, breakatwhitespace=true]

MATCH (m:Measurement)-[:HAS_UNIT]->(u:Unit)
WHERE m.name = "length"
RETURN u.name
\end{lstlisting}

\subsubsection{Логирование и обработка ошибок}

Для мониторинга работы сервера используется система логирования. Основные особенности:

\begin{enumerate}
    \item Логи ошибок записываются в файл \texttt{logs/errors.log}.
    \item Информация о запросах и ответах записывается в консоль.
\end{enumerate}

\subsection{Разработка клиентской части}

\subsubsection{Общая структура фронтенда}

Клиентская часть веб-приложения разработана с использованием фреймворка Vue.js, который обеспечивает удобное построение интерфейсов и модульную организацию кода. В проекте используется Composition API, что позволяет эффективно управлять состоянием приложения.

Основные технологии, применяемые во фронтенде:
\begin{enumerate}
    \item Vue.js — фреймворк для разработки пользовательских интерфейсов.
    \item PrimeVue — библиотека UI-компонентов.
    \item Pinia — инструмент для управления состоянием.
    \item Vue Router — система маршрутизации.
    \item Axios — библиотека для работы с API.
\end{enumerate}

Фронтенд структурирован следующим образом:
\begin{enumerate}
    \item \texttt{src/main.ts} — точка входа в приложение.
    \item \texttt{src/router/index.ts} — маршрутизация внутри приложения.
    \item \texttt{src/stores/} — хранилища состояния (Pinia).
    \item \texttt{src/components/} — простые компоненты.
    \item \texttt{src/api/axios.ts} — конфигурация отправки API-запросов.
    \item \texttt{src/views/} — представления страниц.
    \item \texttt{src/layout/} — компоненты макета.
\end{enumerate}

\subsubsection{Компоненты и представления}

Фронтенд-приложение содержит набор компонентов и представлений, организованных в директориях \texttt{src/components} и \texttt{src/views}. Компоненты обеспечивают переиспользуемые элементы интерфейса, а представления реализуют основные страницы приложения.

\paragraph{Компоненты}

\begin{enumerate}
    \item \texttt{Editor} — компонент редактора текста, используемый для ввода описаний экспериментов.
    \begin{enumerate}[label=\arabic{enumi}.\arabic*.]
        \item Входные параметры: \texttt{description} (текст описания).
        \item Выходные параметры: обновленный текст.
    \end{enumerate}

    \item \texttt{FloatingConfigurator} — компонент конфигуратора, который позволяет изменять настройки отображения данных.
    \begin{enumerate}[label=\arabic{enumi}.\arabic*.]
        \item Входные параметры: \texttt{settings} (объект параметров конфигурации).
        \item Выходные параметры: обновленный объект конфигурации.
    \end{enumerate}

    \item \texttt{Table} — компонент таблицы для отображения результатов измерений.
    \begin{enumerate}[label=\arabic{enumi}.\arabic*.]
        \item Входные параметры: \texttt{columns} (информация о столбцах), \texttt{data} (информация о ячейках).
        \item Выходные параметры: отредактированные данные.
    \end{enumerate}
\end{enumerate}

\paragraph{Представления}

\begin{enumerate}
    \item \texttt{Auth/SignIn} — страница входа в систему.
    \item \texttt{Auth/SignUp} — страница регистрации нового пользователя.
    \item \texttt{Dashboard/CreateExperimentDialog} — диалог создания нового эксперимента.
    \item \texttt{Dashboard/CreateFolderDialog} — диалог создания новой папки.
    \item \texttt{Dashboard/Dashboard} — главная страница панели управления.
    \item \texttt{Dashboard/ExperimentActions} — панель действий над экспериментом (удаление, перемещение и т.д.).
    \item \texttt{Dashboard/ExperimentTable} — таблица всех доступных экспериментов.
    \item \texttt{Dashboard/MoveExperimentDialog} — диалог перемещения эксперимента в другую папку.
    \item \texttt{Editors/LabExperiment} — интерфейс редактирования лабораторного эксперимента.
    \item \texttt{Editors/CompExperiment} — интерфейс редактирования лабораторного эксперимента.
    \item \texttt{Utils/AccessDenied} — страница ошибки доступа.
    \item \texttt{Utils/Error} — страница общей ошибки.
\end{enumerate}

Кроме самописных компонентов, для создания пользовательского интерфейса используются компоненты PrimeVue. В том числе главные элементы интерфейса организованы построены с помощью:

\begin{enumerate}
    \item \texttt{AppLayout} — основной макет приложения, содержащий боковое меню (\texttt{AppSidebar}), навигационную панель (\texttt{AppTopbar}) и футер (подвал) (\texttt{AppFooter}).
    \item \texttt{AppMenu} — навигационное меню, содержащее основные ссылки.
\end{enumerate}

Все эти компоненты и представления работают вместе, формируя клиентскую часть приложения и обеспечивая удобный пользовательский интерфейс.

\subsubsection{Маршрутизация}

Веб-приложение использует систему маршрутизации Vue Router. Ниже приведены основные маршруты:

\begin{enumerate}
    \item \texttt{/} — главная страница (панель управления экспериментами).
    \item \texttt{/auth/accessdenied} — страница ошибки доступа.
    \item \texttt{/auth/error} — страница общей ошибки.
    \item \texttt{/auth/login} — страница входа в систему.
    \item \texttt{/auth/signup} — страница регистрации нового пользователя.
    \item \texttt{/documentation} — страница документации.
    \item \texttt{/experiment/\{id\}} — страница просмотра и редактирования конкретного эксперимента.
\end{enumerate}

Каждый маршрут связан с соответствующим представлением во фронтенде и управляется Vue Router.

\subsubsection{Управление состоянием}

Для управления глобальным состоянием используется Pinia. Этот пакет позволяет:

\begin{enumerate}
    \item Глобально управлять состоянием без необходимости пробрасывания данных через длинную цепочку компонентов.
    \item Обеспечивать предсказуемость данных, поскольку все изменения происходят через строго определённые методы.
    \item Легко организовывать разделение логики в приложении, сохраняя состояние в отдельных модулях.
    \item Хранить данные между разными страницами и компонентами, не теряя их при переходах.
\end{enumerate}

Приложение использует два хранилища:

\begin{enumerate}
    \item \texttt{src/stores/core.ts} — отвечает за глобальные настройки интерфейса и данные пользователя, которые могут потребоваться в любой части приложения.
    \item \texttt{src/stores/dashboard.ts} — хранит данные и методы, согласованный доступ к которым необходим в разных компонентах, составляющих панель управления.
\end{enumerate}

\subsubsection{Взаимодействие с API}

Для работы с API используется библиотека Axios. Библиотека предоставляет удобный механизм перехватчиков (interceptors), позволяющих совершать манипуляции над каждым запросом перед его отправкой или результатом запроса после его завершения. С помощью этого механизма настроены:
\begin{enumerate}
    \item Автоматическое прикрепление access-токена в заголовок \texttt{Authorization}
    \item Автоматическое обновление access-токена, если код ответа равен 401 (Unauthorized)
\end{enumerate}

\subsubsection{Редактирование экспериментов}

Редактирование экспериментов является одной из ключевых функций веб-приложения. В системе предусмотрены отдельные представления для работы с лабораторными и вычислительными экспериментами. Пользователь может изменять описание эксперимента, редактировать параметры и вносить изменения в таблицу результатов.

Редактирование лабораторного эксперимента осуществляется через интерфейс, содержащий текстовый редактор и таблицу данных (представлен на рисунке \ref{pic:lab_experiment_editor}). Пользователь может изменять структуру эксперимента, добавлять или удалять столбцы, а также управлять привязками к онтологиям. 

% Для вычислительных экспериментов предусмотрены дополнительные возможности, такие как редактирование входных данных, параметров модели и контекстной информации. Система автоматически проверяет корректность введённых значений и обеспечивает привязку данных к соответствующим онтологическим сущностям.

Редактирование доступно только владельцам эксперимента, а изменения сохраняются в базе данных и могут быть восстановлены в случае необходимости.

\begin{figure}[H]
    \centering
    \includegraphics[width=\linewidth]{img/experiment_view.png}
    \caption{Окно просмотра данных эксперимента}
    \label{pic:lab_experiment_editor}
\end{figure}
\vspace{0.5cm}

\subsubsection{Таблица результатов измерений}

Таблица результатов измерений является важной частью экспериментов, позволяя пользователю фиксировать и анализировать полученные данные. Каждая колонка таблицы может быть связана с определённой онтологией, что позволяет обеспечить корректную интерпретацию данных.

При создании или редактировании эксперимента пользователь может задать для каждого столбца его тип, единицы измерения и соответствующую онтологическую метку. Это облегчает обмен данными с другими системами. Отображение таблицы реализовано в виде интерактивного компонента, который позволяет пользователям сортировать, фильтровать и редактировать данные.

Привязка к онтологиям делает работу с экспериментами более структурированной и помогает избежать ошибок, связанных с некорректными единицами измерения или несовместимыми форматами данных.

\subsubsection{Адаптивность и мобильная версия}

Интерфейс приложения адаптирован под мобильные устройства с использованием медиазапросов и классов TailwindCSS~\cite{Library:TailwindCSS} (рис. \ref{pic:mobile_version}). Адаптивность обеспечивается настройками макета, темы, акцентного цвета и цвета фона (рис. \ref{pic:other_colors}).

Благодаря этому при уменьшении ширины экрана боковая панель скрывается, а контент центрируется, а пользователи с особенностями зрения могут подобрать комфортное для них сочетание цветов и темы, делающее управляющие элементы сайта наиболее заметными, а работу комфортной.

\begin{figure}[H]
    \centering
    \begin{minipage}{0.45\linewidth}
        \centering
        \includegraphics[width=\linewidth]{img/mobile_version.png}
        \caption{Окно просмотра данных эксперимента}
        \label{pic:mobile_version}
    \end{minipage}\hfill
    \begin{minipage}{0.45\linewidth}
        \centering
        \includegraphics[width=\linewidth]{img/other_colors.png}
        \caption{Выбор других сочетаний цветов и темы}
        \label{pic:other_colors}
    \end{minipage}
\end{figure}
\vspace{0.5cm}

\subsection{Работа с базой данных}

\subsubsection{Структура базы данных и схемы таблиц}

В качестве основной СУБД используется PostgreSQL, обеспечивающая высокую надёжность и поддержку сложных реляционных связей. Схема базы данных (рис. \ref{pic:postgres_scheme}) организована таким образом, чтобы минимизировать дублирование данных и оптимизировать запросы.

\begin{figure}[H]
    \centering
    \includegraphics[width=\linewidth]{img/postgres_scheme.png}
    \caption{Схема базы данных}
    \label{pic:postgres_scheme}
\end{figure}
\vspace{0.5cm}

\subsubsection{ORM-модели}

Для работы с базой данных используется SQLAlchemy, обеспечивающая удобную работу с объектами и связями. Модели представляют таблицы базы данных в виде классов, что позволяет писать гибкий и безопасный код. ORM упрощает работу с базой, делает возможным автоматическое создание миграций для базы данных и сокращает количество SQL-запросов, оптимизируя их.

Основные сущности перечислены в \ref{ORM models}.

\subsubsection{Управление миграциями}

Alembic используется для управления схемой базы данных, упрощая процесс обновления структуры данных без потери информации. Среди возможностей этого инструмента особенно полезными можно назвать автоматическое создание миграций при изменении зарегистрированных моделей, возможность отката к предыдущим версиям схемы, и, как следствие, контроль изменений базы через версионирование. Всё это позволяет плавно изменять схему базы, исключая конфликты данных.

\subsection{Интеграция с онтологиями}

\subsubsection{Способы привязки данных к онтологиям}

Каждый столбец таблицы измерений может быть связан с онтологией (рис. \ref{pic:linked_to_ontology_column}), обеспечивая стандартную интерпретацию данных. Связь осуществляется через таблицу \\
\texttt{ontology\_descriptions}, содержащую идентификатор эксперимента, имя параметра и ссылку на соответствующую сущность в онтологии. Такое автоматизированное сопоставление данных снижает вероятность ошибок при интерпретации.

\begin{figure}[H]
    \centering
    \includegraphics[width=\linewidth]{img/ontology_linking.png}
    \caption{Столбец, привязанный к онтологии}
    \label{pic:linked_to_ontology_column}
\end{figure}
\vspace{0.5cm}

\subsubsection{Поддерживаемые онтологии}

Приложение поддерживает две онтологии:
\begin{enumerate}
    \item OM2 — онтология физических величин и единиц измерения.
    \item СhEBI — онтология химических соединений.
\end{enumerate}

\subsubsection{Возможности расширения онтологической поддержки}

Система спроектирована таким образом, чтобы поддерживать подключение новых онтологий. Добавление новых источников знаний требует лишь обновления базы данных, API и написания модуля со специфичными для онтологии запросами.

% \todo[inline]{Вставить пример работы с онтологиями в интерфейсе}

\subsection{Тестирование и отладка}

\subsubsection{Модульное тестирование серверной части}

Для тестирования серверной части используются unit-тесты (рис. \ref{pic:backend_unit_tests}), написанные с использованием пакета pytest~\cite{Library:Pytest}, проверяющие корректность работы моделей, API и механизмов аутентификации.

\begin{figure}[H]
    \centering
    \includegraphics[width=\linewidth]{img/python_unit_tests.png}
    \caption{Пример запуска unit-тестов серверной части}
    \label{pic:backend_unit_tests}
\end{figure}
\vspace{0.5cm}

\subsubsection{Интеграционное тестирование API}

Для тестирования API применяется класс TestClient из FastAPI, который позволяет проверять работу маршрутов в изолированной среде.

% \subsubsection{Тестирование клиентской части}

% Фронтенд тестируется с помощью Cypress и Jest. Проверяются основные пользовательские сценарии: аутентификация, создание и редактирование экспериментов, работа с таблицами.

% \todo[inline]{Вставить пример теста интерфейса}

\subsection{Развёртывание и эксплуатация}

\subsubsection{Подготовка среды}

Приложение контейнеризовано с использованием Docker. Все компоненты (серверная часть, клиентская часть, базы данных) запускаются через Docker Compose, обеспечивая предсказуемость окружения. Это ползволяет развернуть приложение на любом сервере без сложной настройки. Также предусмотрены различные конфигурации Docker Compose, предназначенные специально для разработки, тестирования и продуктового окружения.

В продуктовом окружении приложение разворачивается с помощью Docker Compose, который управляет следующими сервисами:
\begin{enumerate}
    \item backend — контейнер с серверной частью.
    \item frontend — контейнер с обратным прокси-сервером для работы с клиентом.
    \item db — контейнер с PostgreSQL.
    \item neo4j — контейнер с Neo4j.
\end{enumerate}

\subsubsection{Настройка Nginx как обратного прокси}

Для маршрутизации запросов используется Nginx, который проксирует HTTP-запросы от фронтенда к бэкенду, а также управляет статическими файлами, что повышает общую производительность системы и отзывчивость для пользователя.

Основные настройки:
\begin{enumerate}
    \item Проксирование API-запросов на бэкенд.
    \item Кеширование статических файлов.
\end{enumerate}

\anonsubsection{Выводы по главе}

В данной главе рассмотрена программная реализация веб-приложения, включая серверную и клиентскую части, работу с базой данных, интеграцию с онтологиями, тестирование и развёртывание. Архитектура приложения построена по клиент-серверной модели, обеспечивая высокую производительность и удобство работы с данными. Использование ORM упростило взаимодействие с базой данных, а поддержка онтологий позволила связывать экспериментальные данные с формализованными научными сущностями. Гибкое управление состоянием во фронтенде и система тестирования способствуют надёжности работы интерфейса и API. Развёртывание через контейнеризацию обеспечивает переносимость, а механизмы аутентификации и защиты данных гарантируют безопасность. В результате создана масштабируемая и расширяемая система для управления лабораторными и вычислительными экспериментами, адаптируемая к новым требованиям.