\anonsection{Заключение}

В данной работе была рассмотрена проблема управления данными лабораторных и вычислительных экспериментов и предложено её решение в виде веб-приложения, обеспечивающего удобную работу с экспериментальными данными и их привязку к онтологиям. Был проведён анализ предметной области, который показал, что существующие решения не позволяют эффективно стандартизировать данные и автоматически проверять корректность единиц измерения. Разработка нового инструмента была направлена на устранение этих недостатков за счёт интеграции онтологий, структурированного хранения данных и удобного интерфейса для работы исследователей.

В процессе проектирования была определена архитектура системы, основанная на клиент-серверной модели. Проработаны пользовательские сценарии, продумана структура базы данных и реализованы механизмы взаимодействия между компонентами системы. В качестве ключевых требований были выделены поддержка онтологий, удобство работы с экспериментами и возможность гибкой настройки системы. Разработанное приложение позволяет пользователям фиксировать и редактировать данные, организовывать эксперименты, привязывать измерения к онтологическим сущностям, а также получать удобный доступ к структурированной информации.

В ходе реализации была создана серверная часть, обеспечивающая надёжное хранение данных, обработку API-запросов и механизмы аутентификации. Клиентская часть разработана с учётом удобства использования, что позволяет исследователям легко управлять экспериментами и их параметрами. Интеграция с онтологиями позволяет стандартизировать данные, обеспечивая их корректную интерпретацию и автоматическое сопоставление единиц измерения. Проведённое тестирование подтвердило работоспособность системы, а контейнеризация и автоматизация развертывания обеспечили удобство эксплуатации.

Перспективы дальнейшего развития приложения связаны с добавлением поддержки новых типов экспериментов, расширением возможностей по импорту и экспорту данных в различные форматы, а также интеграцией с внешними базами знаний. Внедрение инструментов для визуализации экспериментальных данных и статистического анализа позволит повысить удобство работы пользователей и сделать приложение более универсальным для различных научных областей.

\newpage
\printbibliography[title=Список источников, heading=bibintoc]
\newpage

\addition{Техническое задание}{TZ}
Представлено отдельным документом <<Техническое задание. Веб-приложение для ведения лабораторных и вычислительных экспериментов с поддержкой онтологий\unskip>>.

 \addition{Руководство оператора}{RO}
 Представлено отдельным документом <<Руководство оператора. Веб-приложение для ведения лабораторных и вычислительных экспериментов с поддержкой онтологий\unskip>>.

\addition{Программа и методика испытаний}{PMI}
Представлено отдельным документом <<Программа и методика испытаний. Веб-приложение для ведения лабораторных и вычислительных экспериментов с поддержкой онтологий\unskip>>.

\addition{Текст программы}{TP}
Представлено отдельным документом <<Текст программы. Веб-приложение для ведения лабораторных и вычислительных экспериментов с поддержкой онтологий\unskip>>.